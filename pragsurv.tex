\documentclass[11pt,twoside]{article}
\usepackage[toc,page,header]{appendix}
\usepackage{pdfpages}
\usepackage{csquotes}
\usepackage{epigraph}
\usepackage{changepage}
\usepackage{fontspec}
\defaultfontfeatures{Scale=MatchLowercase}
\setmainfont[Mapping=tex-text]{Times New Roman}
\setsansfont[Mapping=tex-text]{Arial}
\setmonofont{Courier}

\usepackage{float}
\usepackage{turnstile}
\usepackage{bussproofs}

\usepackage{geometry}
\geometry{letterpaper}

\newtheorem{theorem}{Theorem}
%\newtheorem{cor}{Corollary}
%\newtheorem{lem}{Lemma}
%\theoremstyle{remark}
\newtheorem{remark}{Remark}

\newtheorem{objection}{Objection}
\newenvironment*{response}[1][]{\noindent
\textbf{Response to Objection #1.}
\begin{adjustwidth}{1em}{1em}
}
{\end{adjustwidth}
\vspace{1ex}
}


%\usepackage[parfill]{parskip}    % Activate to begin paragraphs with an empty line rather than an indent

\usepackage{graphicx}
\usepackage[leftcaption]{sidecap}
\sidecaptionvpos{figure}{c}

%\usepackage{amssymb}

\usepackage{epstopdf}
\DeclareGraphicsRule{.tif}{png}{.png}{`convert #1 `dirname #1`/`basename #1 .tif`.png}

\usepackage[
bibstyle=numeric,
citestyle=authortitle,
natbib=true,
hyperref,bibencoding=utf8,backref=true,backend=biber]{biblatex}

\usepackage{hyperref}
\hypersetup{
    bookmarks=true,         % show bookmarks bar?
    unicode=true,          % non-Latin characters in Acrobat’s bookmarks
    pdftoolbar=true,        % show Acrobat’s toolbar?
    pdfmenubar=true,        % show Acrobat’s menu?
    pdffitwindow=false,     % window fit to page when opened
    pdfstartview={FitH},    % fits the width of the page to the window
    pdftitle={Pragmatism and Survey Research},    % title
    pdfauthor={Author},     % author
    pdfsubject={Subject},   % subject of the document
    pdfcreator={Creator},   % creator of the document
    pdfproducer={Producer}, % producer of the document
    pdfkeywords={keyword1} {key2} {key3}, % list of keywords
    pdfnewwindow=true,      % links in new window
    colorlinks=true,       % false: boxed links; true: colored links
    linkcolor=blue,          % color of internal links
    citecolor=blue,        % color of links to bibliography
    filecolor=magenta,      % color of file links
    urlcolor=cyan           % color of external links
}
\usepackage{draftwatermark}


\usepackage{fancyhdr}
\setlength{\headheight}{15.2pt}
\pagestyle{fancy}

\lhead[Pragmatism \& Survey Research]{\thepage}
\chead[]{}
\rhead[\thepage]{Pragmatism \& Survey Research}

\title{Pragmatism and Survey Research}
\author{G. A. Reynolds}
\date{\today}
\bibliography{%
../bib/abstracts.bib,%
../bib/biology.bib,%
../bib/causality.bib,%
../bib/em.bib,%
../bib/logic.bib,%
../bib/mind.bib,%
../bib/philosophy.bib,%
../bib/pragmatism.bib,%
../bib/psychomet.bib%
../bib/psychometrics.bib,%
../bib/misc.bib,%
../bib/measurement.bib,%
../bib/psychology.bib,%
../bib/variables.bib,%
../bib/val.bib,%
../bib/validity.bib,%
}

%% Macros

\newcommand{\SM}{Standard Model}
\newcommand{\XSM}{Extended Standard Model}

\newcommand{\SMeth}{Survey Methodology}

\newcommand{\SR}{Survey Research}
\newcommand{\sr}{survey research}
\newcommand{\SRIV}{Survey Interview}
\newcommand{\sriv}{survey interview}
\newcommand{\SIV}{Survey Interviewing}
\newcommand{\FI}{Field Interviewer}
\newcommand{\Iver}{Interviewer}
\newcommand{\R}{Respondent}
\newcommand{\LPR}{Legal Permanent Resident}
\newcommand{\ART}{Assimilated Response Technique}
\newcommand{\GAM}{Grouped Answer Method}
\newcommand{\IOM}{Instrument of Measurement}

\newcommand{\MIE}{\textit{MIE}}

\includeonly{%
%% pilots,cards
}
%%%%%%%%%%%%%%%%%%%%%%%%%%%%%%%%%%%%%%%%%%%%%%%%%%%%%%%%%%%%%%%%
\begin{document}
\maketitle
\nocite{*}

\begin{abstract}
abstract
\end{abstract}

\tableofcontents
\listoffigures

\newpage
%%%%%%%%%%%%%%%%%%%%
\section{Introduction}

\begin{abstract}
Generally speaking, \sr{} is dominated by what might be called
scientistic cognitivism.  The centrality of cognitivism is clearly
evident in the dominant models of ``the survey process'' and the
practices of ``cognitive interviewing''.  The ``scientistic'' part is
evident in the sort of language that dominates survey methodology,
which routinely treats questionnaires as instruments of measurement
and models interviewing on the experimental methods of the physical
sciences.

Meanwhile a quiet revolution has been underway for the past several
decades in the human sciences (including e.g. AI and neuroscience).

The ``new sciences'' - ``cognitive'' this or that - are generally
speaking neither new nor particularly scientific.  The only genuinely
new element is computation.  The emergence of a well-defined concept
of computability in the first half of the 20th century did indeed mark
a conceptual innovation of truly historic proportions.  But the
various ``cognitive'' sciences to which it gave rise, once scholars
began to take a computational perspective on psychology, were not
revolutionary; they only advanced an agenda that has its roots in the
17th century Enlightenment.  Those ``cognitive'' sciences in which
computation plays a central role seek to mathematicize the human, just
as Galilleo, Newton, and other Enlightenment scientists have sought to
mathematicize nature.

The truly revolutionary movement is marked by the development of
Pragmatism.  Properly understood, Pragmatism doesn't advance the
agenda of the first Enlightenment; it turns it upside down.

Pragmatism liberates us from the tyranny of objective reality (Truth,
etc.) without stranding us in a jejune relativism.  It cheerfully
accepts the existence of the real world and the constraints it imposes
on us, but it rejects the notion that we can somehow find the
Archimedean point of purchase that will allow us to prize apart the
real from the apparent, the True from the False.  It denies that there
is any one true method, scientific or otherwise, that will lead us to
the promised land of True Knowledge.  It denies that we can learn to
speak the one true language of nature, or that such a language even
exists.  
\begin{remark}
But it also denies, for all that, that we are condemned to ignorance
and error.  It insists that we can learn, that we can cope with each
other and our environments in the ways that matter.  Etc.  TODO: state
the positive case in a way that ties it back to our ordinary
intuitions of truth, objectivity, etc.
\end{remark}

Up to now, however, the Pragmatist Enlightenment has had relatively
little impact on \sr{}.  Even the relatively small number of sr{}
researchers who have tried (since at least the early 90s) to draw
attention to the contextual and interactive aspects of survey
interviewing have tended to accept the main commitments of the
traditional cognitivist perspective.  They tend to treat context and
interactivity as important but essentially peripheral aspects of a
``process'' whose center remains firmly entangled by cognitivist and
representationalist commitments: to mental entities and processes, the
autonomy of language, the atomicity of words, representational
semantics, and so forth.

The \sr{} literature shows distinct signs of a cargo-cult science
mentality.  A clear example is the use of the term ``probe'' in
discussions of cognitive interviewing.  The metaphor is obvious: a
probe is a scientific instrument used to examine a specimen.  So long
as this is treated as nothing more than a metaphor there is no
problem; but the ``theory'' of cognitive interviewing tends to take it
much farther.  It takes the notion of a probe literally, and construes
ordinary questions as scientific instruments designed to probe the
cognitive architecture of responding subjects.  The clear implication
is that there is something distinctive about the ``probes'' used in
cognitive interviewing, something that makes them scientific
instruments, when in fact they are nothing more than ordinary
discursive performances.  Merely calling a follow-up question a
``probe'' does make it an instrument; still less does it make it
``scientific''.  To pretend otherwise is to engage in cargo-cult
science.  To put it another way: the cognitivist theory under which we
are to treat some verbal performances as ``probes'' is the
\textit{only} justification we have for thinking they are scientific
instruments.  But it provides no means of distinguishing a set of such
performances from any other discursive performance, no way of deciding
what counts as a probe, other than its own theoretical claims.  The
logic is entirely circular.

Critical v. constructive

The purpose of this paper is two-fold.  First it provides an overview
of contemporary Pragmatism, in order to give substance to the claim
advanced above as to its revolutionary character.  One way to do this
is by contrasting it with its opposite number(s), so the result will
be to expose and clarify the fundamental themes and commitments of
representationalism and cognitivism, etc.  This will sharpen some
boundaries.

Once the background issues - theoretical, methodological, philosophical
- are clear, the other task is to examine, at least in a preliminary
manner, the implications of Pragmatism for \SR{}.  What is on offer is
a radical re-conceptualization of the entire enterprise.  A move away
from scientism toward a more properly (and appropriately)
anthropological perspective.

Caveat: we are not talking here about merely methodological issues.
It is not a question of doing the same thing, only better; of finding
an innovative method that solves the old problems.  What is suggested
is rather a fundamental change in the way we conceive of the task,
asking different questions, discarding the old questions as not useful
or even very meaningful.

\end{abstract}

\section{\SR{}: The Received View}

Field v. experimental science \parencite{ryan_replication_2011},
\parencite{hurlbert_pseudoreplication_1984}

\begin{remark}
What is ``\SR{}''?  Two answers: research that \textit{uses} surveys
to collect (and construct) data, and research into the nature of
surveys.  First, each survey project studies something, or several
things (each question being ``about'' something).  Second, \SR{} as a
kind of meta-discipline studies surveys; usually this goes by
``\SMeth{}''.
\end{remark}

\begin{itemize}
\item \SR{}: production and collection of social science ``data''
  (more accurately: \textit{facta}, mades, rather than \textit{data},
  givens) by means of questions.  Better: survey \textit{based}
  research; research that \textit{uses} survey techniques to enable
  study of a phenomenon.  How then do we know that these techniques
  are appropriate for the object of study?
\item \SMeth: study of the use of questions to produce and collect data
\end{itemize}

\subsection{The \SM{}}

Laboratory model: based entirely on (bad) analogy to the physical
sciences.

\subsection{The \XSM{}}

XSM = SM plus interaction

\subsection{Survey Methodology}

Monism v. pluralism.

%%%%%%%%%%%%%%%%
\section{Pragmatism}

\epigraph{Consider what effects, that might conceivably have practical bearings, we conceive the object of our conception to have. Then our conception of these effects is the whole of our conception of the object.}
{CS Peirce (CP5.402)}

``Although the term pragmatism is frequently used to characterize some or other highly specific thesis or program, pragmatism is not and never was a school of thought unified around a distinctive doctrine.'' talisse 1

``As a kind of naturalism, pragmatism is partly a thesis about the
relation of philosophy to the natural sciences; consequently, one
should expect pragmatists to engage the questions of the proper aims
and methods of philosophy.'' talisse 9

``In its most muscular form, the pragmatist thesis is that, once we understand properly the nature of philosophy, we will discover that there are no philosophical problems anyway.'' talisse 9

Pragmatism integrates naturalism (science) and humanism:

``What makes each of these authors pragmatist is their emphasis on
naturalistic and variously humanistic accounts of philosophical
problems and solutions. One of the reasons as to the variety of
pragmatisms is the variety of humanisms available to pragmatists.''
talisse 5
\subsection{Major Themes}

Negative and positive.

\begin{itemize}
\item anti-foundationalism
\item anti-representationalism
\item normativity
\item inferentialism
\item expressivism
\item cognitivism, mentalism
\item naturalism
\item evolution \& statistics
\end{itemize}

``Quine’s corpus presents an ongoing development of a few key
pragmatist and naturalist in- sights about science, language, and
ontology, and an attempt to fit them together. Importantly, Quine
proceeds by way of critical engagement with nonnaturalist critics and
interlocutors....the case for pragmatism was to be made on a
case-by-case basis, not by way of a comprehensive philosophical
system.'' Talisse intro p. 8-9

\subsection{The Strategy}

Brandom, Price, etc. adopt similar analytic/explanatory strategies which have
their roots in Peirce's Maxim.

``Roughly speaking, deflationists suggest that semantic vocabulary enables speakers to do useful things with (other, pre-existing) words and sentences - to do things which they couldn't do so well, or at all, without semantic vocabulary...A functional account of this kind is, inter alia, an account of the use speakers make of the semantic vocabulary concerned. It explains the vocabulary in terms of its use and function in the linguistic community. But it does not reduce or analyse facts about meaning to facts about use. Instead it explains talk of meanings, and tells us what it takes to belong to a community who go in for such talk.'' Price, Defl about truth p. 112

``If semantic properties do attach to physical objects in a primary sense, then deflationism is is a non-starter. In particular, it is not enough to try to show that these philosophers are looking for the wrong sort of property - a thick notion of aboutness, where a thin one would do, for example. As deflationists, we need to argue that they looking in the wrong place, that they have the wrong conception of the nature of the problem.'' Price 112

``The solution, I think, is to abandon the idea that among the goals of a use-based theory of meaning should be that of providing a non-semantic reduction of propositions of the form "x means F". On the contrary, I think, the right approach to these locutions is that applied with such success by deflationists in the case of truth: viz. to explain the function of such a locution - in general, the function of talk about meaning - in terms which don't require that it refers to substantial properties.

As noted above, such an approach is bound to appeal to facts about usage. It will tell us under what circumstances speakers use the locutions concerned, and what functions this use serves in the speech communities concerned. But instead of analysing facts about meaning in terms of facts about use, it explains our talk of meanings, and tells us what habits of usage underlie such a discourse.'' same, p. 115


\subsection{Brandom}

Contemporary philosophical pragmatism receives its most complete and
thorough exposition in Robert Brandom's masterpiece \enquote{Making It
  Explicit}\footnote{\MIE{} is over 600 pages of close argument covering most of the philosophical topics of interest to \SR{}.  For a more manageable introduction to Brandom's ideas see his \parencite{brandom_precis_1997}, \parencite{brandom_articulating_2001}.  See also \parencite{brandom_perspectives_2011}, \parencite{brandom_reason_2009}, and \parencite{brandom_between_2008}.}.

\subsubsection{Sellars: Myth of the Given, Space of Reasons}

\subsubsection{Sellars: Language Entries}

This is the device that accounts for the relation of causal and
rational orders.  It is true that the world in some sense has a causal
influence on our language performances, but that is not enough to
account for the intelligibility of those performances.  When we
declare ``That's red'' in the presence of red things, we do so
``because'' (in some sense) of those red things and their (causal)
relation to us.  This is what Sellars dubbed a ``language entry''
move.  But that sort of causality cannot account for the conceptual
content of our utterance.

\subsubsection{Brandom: From Sentience to Sapience}

To say ``That's red'' is to apply the \textit{concept} ``red'', and
the subpersonal, causal relation between the presence of a red thing
and our conceptually contentful utterance cannot account for this.  It
cannot account for our ability to apply the concept red
\textit{correctly}, to red things, not non-red things.  After all, if
the presence of red things caused us to say ``That's red'', then we
would in fact say that hundreds or thousands of times a day.  A causal
model cannot account for four fundamental normative aspects of our
behavior: the ability to lie, to err, to hedge (``It \textit{seems}
red''), and to remain silent.

Brandom's Parrot: one of Brandom's favored illustrative examples is a
parrot trained to squawk ``That's red'' in the presence of red things.
This is an example of \textit{sentience} rather than
\textit{sapience}.  Brandom's Parrot is not sapient; its performance
does not count as conceptually contentful (rational), since it does
not involve the application of concepts.  This is where inferential
semantics enter the picture: the content of ``red'' is essentially
inferentially articulated.  To count as a concept user the parrot must
be capable of drawing inferences (either explicitly or implicitly)
involving the concept ``red''.  For example, it must know that
``That's green'' is incompatible with ``That's red''.  Those
inferences, in turn, are only intelligible in terms of what Brandom
(following Sellars) calls ``the game of giving and asking for
reasons''.

Question-based interviews: only intelligible as ``language games'',
denizens of the Space of Reasons.

\begin{remark}
  The fundamental mistake made by the \SM{} is failure to distinguish
  between distinct ``orders of explanation'': the subpersonal, causal
  world, and the personal, discursive, rational world.  Q\&A-based
  interviewing lives in the latter, not the former.  The notion that
  questions are stimuli that ``cause'' responses is fundamentally
  mistaken.  Whatever causal relations may obtain between a question
  utterance and the ensuing response utterance are not relevant to the
  intelligibility of the game.  Responses have \textit{reasons}, not
  causes.
\end{remark}

\begin{remark}
  An example would be useful here.  Maybe ``How old are you?''  A
  correct response to this question is one that involves propositional
  commitments and entitlements.  It does not involve any causal
  relationship to the question, still less to any ``latent'' age
  variable whose value is, say ``27 years''.  Crudely put, you know
  you're 27 years old if you know that last year you were 26.  More
  accurately, you know \textit{how} to respond because you know the
  rules of the language game, which involves also counting years and
  birthdays.  Consider how children learn their ages: they learn that
  certain verbal performances (e.g. ``I'm four'') are correct,
  regardless of whether they understand what they mean, and they learn
  that every year they have a ``birthday'', after which a different
  performance (``I'm five'') is correct.
\end{remark}

\subsection{Normativity}

The Space of Reasons is a normative space of reasons, not a natural
(causal) space of laws.

\subsection{Vocabularies}

Measurement as description.  Description v. evaluation.  Price on
naturalisms.  The bifurcation thesis.


\subsection{Bibliography}

\noindent
\cite{bacon_pragmatism:_2012} \\
\cite{barnes_ethnomethodology_1985} \\
\cite{baert_pragmatism_2003} \\
\cite{baert_realism_2003} \\
\cite{baert_pragmatism_2004} \\
\cite{baert_towards_2005} \\
\cite{baert_philosophy_2005} \\
\cite{berard_rethinking_2005} \\
\cite{bloor_wittgenstein_2001} \\
\cite{blackburn_invited_1986} \\
\cite{blackburn_steps_2010} \\
\cite{brandom_mie} \\
\cite{brandom_precis_1997} \\
\cite{brandom_articulating_2001} \\
\cite{brandom_pragmatist_2004} \\
\cite{brandom_between_2008} \\
\cite{brandom_reason_2009} \\
\cite{brandom_perspectives_2011} \\
\cite{brandom_analyzing_2011} \\
\cite{brandom_classical_2011} \\
\cite{brandom_vocabularies_2011} \\
\cite{brandom_social_1993} \\
\cite{button_ethnomethodology_1991} \\
\cite{churchill_ethnomethodology_1971} \\
\cite{descombes_minds_2001} \\
\cite{emirbayer_pragmatism_2010} \\
\cite{garfinkel_studies_1984} \\
\cite{garfinkel_ethnomethodologys_2002} \\
\cite{heritage_garfinkel_1984} \\
\cite{kraut_varieties_1990} \\
\cite{loeffler_neo-pragmatist_2009} \\
\cite{lynch_ethnomethodology_2001} \\
\cite{lynch_cognitive_2006} \\
\cite{macdonald_nature_1992} \\
\cite{margolis_reinventing_2002} \\
\cite{margolis_pragmatism_2007} \\
\cite{maynard_diversity_1991} \\
\cite{maynard_toward_2000} \\
\cite{price_true_question_1983} \\
\cite{price_expressivism_2013} \\
\cite{price_naturalism_2013} \\
\cite{price_pluralism_2013} \\
\cite{price_two_2013} \\
\cite{putnam_representation_1991} \\
\cite{putnam_collapse_2002} \\
\cite{putnam_three_2009}
\cite{rorty_method_1981} \\
\cite{rorty_representation_1988} \\
\cite{rorty_PMN} \\
\cite{schatzki_practice_2001} \\
\cite{sellars_empiricism_1997} \\
\cite{tate_foucault_2007} \\
\cite{weiss_reading_2009} \\
\cite{winship_ethnomethodology_2010} \\
\cite{zimmerman_review_1994}

%%%%%%%%%%%%%%%%%%%%
\section{Science}

\begin{remark}
  Q: What does Pragmatism have to say about science, and why should we
  care?  A: Philosophy as therapy (Rorty) or edification
  (Wittgenstein?).  Exposure of unexamined presuppositions and
  consequences, etc.
\end{remark}

\begin{remark}
  The major point: the Pragmatic Enlightenment dethroned
  ``objectivity'' or ``Reality'' as the source of unimpeachable
  external authority over our intellectual lives, just as the 17th
  century Enlightenment remove religion as the source of authority
  over our political and civic lives.  It follows that pragmatism
  undermines science's self image as the one true objective external
  source of authoritative knowledge.  For pragmatism, science is one
  of many human vocabularies, with no legitimate claim to special
  authority as an external, independent arbiter of truth claims.
\end{remark}

\begin{remark}
  This has special significance for the human sciences.  Among other
  things, it suggests social scientists should stop trying to mimic
  physics.  Not for the traditional reason (i.e. that this is
  impossible), but because physics should not be granted such special
  status.  Why should we take physics as the model science?
  Undoubtedly because it is so successful in practice, as predicting
  and manipulating the world.  But practical success is not the same
  as epistemic authority.

  In fact it seems we are seeing a shift in the relative prestige of
  scientific fields; these days biology seems to be displacing physics
  as the model science.
\end{remark}

\subsection{From Empiricism to Pragmatism}

Sellars, Quine, Wittgenstein and the demolition of Empiricism.

Empiricism smuggles the conceptual into perception.  Pragmatism
remedies this by first recognizing that ``all perceptual awareness is
conceptual'' (Sellars, somewhere), and second, that the conceptual is
fundamentally pragmatic (and inferential).

With respect to measurement, this means that the appeal to
isomorphisms between mathematical and ``empirical structures'' is
problematic.  Empirical structures are already conceptual.  They are
not ``given''.  Pragmatism shifts the focus to the practices in virtue
of which we are able to recognize the empirical as such in the first
place.  So the isomorphism of measurement theory must be supplemented
by an account of the relation of the conceptual structure of empirical
systems to the world.  And this relation is essentially pragmatic, a
matter of what we do, how we interact with our external environment,
rather than an antecedently established correspondence between our
concepts and the world.

This can be illustrated in the history of temperature measurement,
where theory and its relation to measurement practice played the
decisive role.

\subsection{Price: Global Expressivism}

From Rorty to Price: science as one among many vocabs, with no special
claim to authority.

\subsection{Measurement}

\begin{abstract}
Measurement pragmatism.  No representation needed.
\end{abstract}

\subsection{False Dichotomies: quantitative v. qualitative ``variables''}



%%%%%%%%%%%%%%%%%%%%
\subsection{Deflating Validity}

\begin{abstract}
Semantic and metaphysical deflationism works as well for validity as
it does for truth and reference.
\end{abstract}

\begin{remark}
  Deflationism seems to depend essentially on some form of
  expressivism.  Or maybe they amount to the same thing?
\end{remark}


%%%%%%%%%%%%%%%%%%%%%%%%
\subsection{Causality and the Space of Reasons}

\begin{abstract}
abstract
\end{abstract}

\noindent
\cite{abell_narrative_2004} \\
\cite{crane_mental_1995} \\
\cite{gross_pragmatist_2009} \\
\cite{jackson_mental_1996} \\
\cite{lowe_causal_1993} \\
\cite{lowe_non-cartesian_2006} \\
\cite{macdonald_mental_1986} \\
\cite{menzies_causation_1993} \\
\cite{morris_causes_1986} \\
\cite{williamson_broadness_1998}

\subsubsection{Conflation of Causal and Logical Relations}


\subsection{False Dichotomies}

Reality-appearance; true-false; etc.

\subsubsection{Analytic-Synthetic (Quine)}
\subsubsection{Fact-Value (Putnam)}
\subsubsection{Qualitative-Quanitative}
\subsubsection{Word-World}

\subsection{Hypothetical Entities}

\subsection{Personal v. Subpersonal}


\subsection{Spaces}

\subsubsection{Natural space of causes}

\subsubsection{Discursive space of reasons}

%%%%%%%%%%%%%%%%
\section{Reconciliation: A Pragmatic Model of Survey Research}

\subsection{The Deontic Scorekeeping Model of Discursive Practice and \SR{}}

\begin{abstract}
Why the deontic scorekeeping model is preferable to others, esp. the
cognitive model.
\end{abstract}

\begin{remark}
  It's a model of discursive, that is rational, practice.  Contrast
  this with most models on offer which tend to focus on subpersonal
  processes; hence the prevalence of talk about ``the survey
  process'', the ``response process'', etc.
\end{remark}

\subsection{A Quality Assurance Model for \SR{}}

\begin{abstract}
abstract
\end{abstract}

%%%%%%%%%%%%%%%%
\section{Notes}

\subsection{Evolution}

Instead of "the QA process", the proper object of investigation is the
local evolution of discourse.

EM studies local produced order.  It may come up with a structural
description.  But locally produced order is the outcome of an
essentially evolutionary process - the mutual adaptation of the
participants to each other and the context.  Also, any such model may
not (probably will not) generalize.  But what does generalize is the
evolutionary mechanism itself, just like in biology.

Rational selection as the mechanism of the evolution of discursive
performances.  What accounts for the deontic attitudes we adopt
regarding performances?  Brandom's account describes the architecture
of such posturings and the significances the institute.  But it does
not really address the logic of discourse as an evolutionary process.

The idea is that Brandom provides an account of discourse qua rational
action.  Different attitudes are endorsed or undertaken for reasons -
that is the source or ground of the intelligibility of discursive
practice.  So if we view the unfolding of discourse as being governed
by the logic of evolution, we can treat Brandom's sort of rational
pragmatism as the selection mechanism that accounts for why some
attitudes (meanings) survive (are endorsed) and others do not.
Meanings that survive must fit into the space of reasons - they must
be assertable and justifiable, even if the participants are unable to
explicitly articulate this.  This makes the evolution of discourse
intelligible as a rational process, rather than a natural process.
Responses to questions are not explicable as effects caused by "true
values" or the like; this would make them fundamentally non-rational.
Or to borrow a bon mot from Garfinkel, this would make respondents
"rational dopes".

Similar language: "negotiation", e.g. "...I suggest that the content
of talk indicates that imposed hierarchies are continually
re-negotiated..."  Negotiation as rational evolution?

The "true score" and other orthodox models account for sentience, not
sapience.

\subsection{Verum Factum}

Cartesianism (spectator, etc.) inspection, discovery, certainty,
foundationism (external foundation grounding knowledge) v.

Verum Factum, geneological/historical, following growth/development,
not certainty but ???; no foundationism, no priviledged vocab, no
external source of authority

Critical notions: authority.  For evidence etc. key idea is authority - the only
kind of authority is the kind we assent to.  So the question is what
do we treat as authoritative and why, rather than how can we discover
the One True external foundational source of authority and learn to
speak its language

Critical notions: vocabulary.  Regardless of what there is, we can
only talk about it by using vocabs.

Relevance to SR: we make our truths, by engaging in dialog with
respondents in order to teach/train them to understand what we want.
In other words we work to make our scorecards converge.  We can never
be sure that researchers and respondents understand each other, have
the same interpretations of qx text, etc.  But we can do what nature
does in evolution and learning: institute a cyclic process of
experiment, feedback, and correction.  This is operational even at the
most simple and basic level of communication.  So we can use this fact
to our advantage.

Communication interactions as not essentially different from processes
of evolution and learning.  Evolutionary process tend to coordinate
organism and environment; learning processes adapt the learner to the
task environment, etc.  Any discursive exchange - even simple
greetings, etc. - does the same sort of thing: coordinate and mutually
adjust the parties to the exchange.

\subsection{Rational Evidence}

Evidence-Based Rational SR

RCT: isolate the causal factor that links Treatment to Outcome

THe mistake make by orthodox SR (shown by its vocab of measurement,
error, etc.) is that it confuses the space of causes and the space of
reasons.

In RCT, we observe a stimulus followed by a response (T followed by O)
and postulate a causal relation.  In SR, we observe a Q performance
followed by a R performance.  In fact this is an idealization since Q
and R cannot be isolated - they are both joint performances.  Ignore
that for now; the point is that what makes them intelligible as
performances is the space of reasons, not causes.  That is, as
discursive episodes they are essentially rational in a way the T-O
trials are not.  By definition, "rational" means involving concepts.
Stimulus-response does not involve concepts and so is not rational in
this favored sense.  The natural world may be lawful, but it is not
rational.

So SR should abandon the orthodox vocab of measurment, etc. in favor
of one involving rationality.  What would "evidence-based" mean, then?
Not the kind of evidence involve in natural science, since such
evidence does not involve concepts and thus meaning.  Instead evidence
inescapably involves meaning and understanding.  What counts as
evidence is what we count as a rational explanation or story.  And
this necessarily involves the perspective of the participants - it is
their rationality, their giving and asking for reasons, that provides
the observational basis of evidence.

 One consequence: Qx does not involve measurement.  SR can use stats
 to statistically measure the collected data, but that is quite
 separate from whether the data measure anything.  So you can say that
 x\% of resondents pick option X, but that does not mean that you have
 measured the distribution of "true values" of some latent variable.
 What you have measure is a distribution of deontic scores, or
 discursive postures.  There is no warrant for claiming that each
 member of the x\% means the same thing by picking X.

\subsection{Misc}

1.  What is a question?  Better: what counts as a question, what is it to ask a question?

2.  Ditto for answer.

Q and A as parts of a whole (holistic view)

Q token v. Q performance, etc.

\subsection{Erotetic Discursive Practice}


EDP as production of data rather than discovery of truth

\subsection{Replication}

Goal is replication.  Compare: blood work, e.g. measuring
cholesteral.  The measuring apparatus reacts to the sample, not the
other way around.  For EDP, respondent reacts to the question, so the
question is analogous to the blood sample.  The response is a kind of
measurement of the question, not the other way around.

Replicability means same setup, same experimental conditions; in EDP
this means replication of conceptual structure, which is accomplished
by the dialog preceding the question.  Traditionally, "ask the same
question"; in practice this is impossible, since what counts is not
the question text but respondent's grasp of the sense.  So the
"experimental setup" should be viewed as the work of teaching the
respondent what the sense of the question is.  Survey interviewing is
essentially interventionist, but this is not necessarily a bad thing,
since lab experiments are too - they "intervene" to set up
experimental "initial conditions".  The difference is that setting up
initial conditions ("same meaning") in question asking means tutoring
the respondent.

\subsection{Myths and Mythologies}

\begin{itemize}
\item The Myth of Question Independence says that the meaning of a
    question is independent of context.  But the meaning of a question
    is always dependent on what came before it.
\item Myth of Autonomy. Interviewer and Respondent.
\item Myth of Error
\end{itemize}

\subsection{Dopes}

Garfinkel's dopes - cultural, judgmental, psychological

Dehumanization.  Orthodox Survey Research (OSR) dehumanizes
participants.  The R is a sampling unit.  The mythology of OSR
measurement treats the human R as a natural object to be measured
rather than a person.


\clearpage
\appendix
\begin{appendices}
\section{Bibliography}
%% \addcontentsline{toc}{chapter}{Bibliography}
%% \bibliographystyle{plainnat}
\printbibliography[heading=none]
\end{appendices}

\end{document}
