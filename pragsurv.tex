\documentclass[11pt,twoside]{article}
\usepackage[toc,page,header]{appendix}
\usepackage{pdfpages}
\usepackage{csquotes}
\usepackage{epigraph}
\usepackage{changepage}
\usepackage{fontspec}
\defaultfontfeatures{Scale=MatchLowercase}
\setmainfont[Mapping=tex-text]{Times New Roman}
\setsansfont[Mapping=tex-text]{Arial}
\setmonofont{Courier}

\usepackage{float}
\usepackage{turnstile}
\usepackage{bussproofs}

\usepackage{geometry}
\geometry{letterpaper}

\newtheorem{theorem}{Theorem}
%\newtheorem{cor}{Corollary}
%\newtheorem{lem}{Lemma}
%\theoremstyle{remark}
\newtheorem{remark}{Remark}

\newtheorem{objection}{Objection}
\newenvironment*{response}[1][]{\noindent
\textbf{Response to Objection #1.}
\begin{adjustwidth}{1em}{1em}
}
{\end{adjustwidth}
\vspace{1ex}
}


%\usepackage[parfill]{parskip}    % Activate to begin paragraphs with an empty line rather than an indent

\usepackage{graphicx}
\usepackage[leftcaption]{sidecap}
\sidecaptionvpos{figure}{c}

%\usepackage{amssymb}

\usepackage{epstopdf}
\DeclareGraphicsRule{.tif}{png}{.png}{`convert #1 `dirname #1`/`basename #1 .tif`.png}

\usepackage[
bibstyle=reading,%%numeric,
entrykey=false,
citestyle=authortitle,
natbib=true,
hyperref,
bibencoding=utf8,backref=true,backend=biber]{biblatex}

\usepackage{hyperref}
\hypersetup{
    bookmarks=true,         % show bookmarks bar?
    unicode=true,          % non-Latin characters in Acrobat’s bookmarks
    pdftoolbar=true,        % show Acrobat’s toolbar?
    pdfmenubar=true,        % show Acrobat’s menu?
    pdffitwindow=false,     % window fit to page when opened
    pdfstartview={FitH},    % fits the width of the page to the window
    pdftitle={Pragmatism and Survey Research},    % title
    pdfauthor={Author},     % author
    pdfsubject={Subject},   % subject of the document
    pdfcreator={Creator},   % creator of the document
    pdfproducer={Producer}, % producer of the document
    pdfkeywords={keyword1} {key2} {key3}, % list of keywords
    pdfnewwindow=true,      % links in new window
    colorlinks=true,       % false: boxed links; true: colored links
    linkcolor=blue,          % color of internal links
    citecolor=blue,        % color of links to bibliography
    filecolor=magenta,      % color of file links
    urlcolor=cyan           % color of external links
}
\usepackage{draftwatermark}


\usepackage{fancyhdr}
\setlength{\headheight}{15.2pt}
\pagestyle{fancy}

\lhead[Pragmatism \& Survey Research]{\thepage}
\chead[]{}
\rhead[\thepage]{Pragmatism \& Survey Research}

\title{Pragmatism and Survey Research}
\author{G. A. Reynolds}
\date{\today}
\bibliography{%
../bib/abstracts.bib,%
../bib/anthro.bib,%
../bib/biology.bib,%
../bib/brandom.bib,%
../bib/causality.bib,%
../bib/cognitivism.bib,%
../bib/embodiment.bib,%
../bib/em.bib,%
../bib/logic.bib,%
../bib/mind.bib,%
../bib/philosophy.bib,%
../bib/pragmatism.bib,%
../bib/psychomet.bib%
../bib/psychometrics.bib,%
../bib/misc.bib,%
../bib/measurement.bib,%
../bib/psychology.bib,%
../bib/science.bib,%
../bib/surveyresearch.bib,%
../bib/variables.bib,%
../bib/val.bib,%
../bib/validity.bib,%
}

%% Macros

\newcommand{\SM}{Standard Model}
\newcommand{\XSM}{Extended Standard Model}

\newcommand{\SMeth}{Survey Methodology}

\newcommand{\SR}{Survey Research}
\newcommand{\sr}{survey research}
\newcommand{\SRIV}{Survey Interview}
\newcommand{\sriv}{survey interview}
\newcommand{\SIV}{Survey Interviewing}
\newcommand{\FI}{Field Interviewer}
\newcommand{\Iver}{Interviewer}
\newcommand{\R}{Respondent}
\newcommand{\LPR}{Legal Permanent Resident}
\newcommand{\ART}{Assimilated Response Technique}
\newcommand{\GAM}{Grouped Answer Method}
\newcommand{\IOM}{Instrument of Measurement}

\newcommand{\MIE}{\textit{MIE}}

\includeonly{%
%% pilots,cards
}

\setlength{\epigraphwidth}{4in}

%%%%%%%%%%%%%%%%%%%%%%%%%%%%%%%%%%%%%%%%%%%%%%%%%%%%%%%%%%%%%%%%
\begin{document}
\maketitle
\nocite{*}

\begin{abstract}
abstract
\end{abstract}

\tableofcontents
\listoffigures

\newpage
%%%%%%%%%%%%%%%%%%%%
\section{Introduction}

\begin{abstract}
Generally speaking, \sr{} is dominated by what might be called
scientistic cognitivism.  The centrality of cognitivism is clearly
evident in the dominant models of ``the survey process'' and the
practices of ``cognitive interviewing''.  The ``scientistic'' part is
evident in the sort of language that dominates survey methodology,
which routinely treats questionnaires as instruments of measurement
and models interviewing on the experimental methods of the physical
sciences.

Meanwhile a quiet revolution has been underway for the past several
decades in the human sciences (including e.g. AI and neuroscience).

The ``new sciences'' - ``cognitive'' this or that - are generally
speaking neither new nor particularly scientific.  The only genuinely
new element is computation.  The emergence of a well-defined concept
of computability in the first half of the 20th century did indeed mark
a conceptual innovation of truly historic proportions.  But the
various ``cognitive'' sciences to which it gave rise, once scholars
began to take a computational perspective on psychology, were not
revolutionary; they only advanced an agenda that has its roots in the
17th century Enlightenment.  Those ``cognitive'' sciences in which
computation plays a central role seek to mathematicize the human, just
as Galilleo, Newton, and other Enlightenment scientists have sought to
mathematicize nature.

The truly revolutionary movement is marked by the development of
Pragmatism.  Properly understood, Pragmatism doesn't advance the
agenda of the first Enlightenment; it turns it upside down.

Pragmatism liberates us from the tyranny of objective reality (Truth,
etc.) without stranding us in a jejune relativism.  It cheerfully
accepts the existence of the real world and the constraints it imposes
on us, but it rejects the notion that we can somehow find the
Archimedean point of purchase that will allow us to prize apart the
real from the apparent, the True from the False.  It denies that there
is any one true method, scientific or otherwise, that will lead us to
the promised land of True Knowledge.  It denies that we can learn to
speak the one true language of nature, or that such a language even
exists.  
\begin{remark}
But it also denies, for all that, that we are condemned to ignorance
and error.  It insists that we can learn, that we can cope with each
other and our environments in the ways that matter.  Etc.  TODO: state
the positive case in a way that ties it back to our ordinary
intuitions of truth, objectivity, etc.
\end{remark}

Up to now, however, the Pragmatist Enlightenment has had relatively
little impact on \sr{}.  Even the relatively small number of sr{}
researchers who have tried (since at least the early 90s) to draw
attention to the contextual and interactive aspects of survey
interviewing have tended to accept the main commitments of the
traditional cognitivist perspective.  They tend to treat context and
interactivity as important but essentially peripheral aspects of a
``process'' whose center remains firmly entangled by cognitivist and
representationalist commitments: to mental entities and processes, the
autonomy of language, the atomicity of words, representational
semantics, and so forth.

The \sr{} literature shows distinct signs of a cargo-cult science
mentality.  A clear example is the use of the term ``probe'' in
discussions of cognitive interviewing.  The metaphor is obvious: a
probe is a scientific instrument used to examine a specimen.  So long
as this is treated as nothing more than a metaphor there is no
problem; but the ``theory'' of cognitive interviewing tends to take it
much farther.  It takes the notion of a probe literally, and construes
ordinary questions as scientific instruments designed to probe the
cognitive architecture of responding subjects.  The clear implication
is that there is something distinctive about the ``probes'' used in
cognitive interviewing, something that makes them scientific
instruments, when in fact they are nothing more than ordinary
discursive performances.  Merely calling a follow-up question a
``probe'' does make it an instrument; still less does it make it
``scientific''.  To pretend otherwise is to engage in cargo-cult
science.  To put it another way: the cognitivist theory under which we
are to treat some verbal performances as ``probes'' is the
\textit{only} justification we have for thinking they are scientific
instruments.  But it provides no means of distinguishing a set of such
performances from any other discursive performance, no way of deciding
what counts as a probe, other than its own theoretical claims.  The
logic is entirely circular.

Critical v. constructive

The purpose of this paper is two-fold.  First it provides an overview
of contemporary Pragmatism, in order to give substance to the claim
advanced above as to its revolutionary character.  One way to do this
is by contrasting it with its opposite number(s), so the result will
be to expose and clarify the fundamental themes and commitments of
representationalism and cognitivism, etc.  This will sharpen some
boundaries.

Once the background issues - theoretical, methodological, philosophical
- are clear, the other task is to examine, at least in a preliminary
manner, the implications of Pragmatism for \SR{}.  What is on offer is
a radical re-conceptualization of the entire enterprise.  A move away
from scientism toward a more properly (and appropriately)
anthropological perspective.

Caveat: we are not talking here about merely methodological issues.
It is not a question of doing the same thing, only better; of finding
an innovative method that solves the old problems.  What is suggested
is rather a fundamental change in the way we conceive of the task,
asking different questions, discarding the old questions as not useful
or even very meaningful.

\end{abstract}

\begin{remark}
  Readers familiar with Peter Winch's 1958 classic
  \citetitle{winch_idea_1958} will recognize many of his themes in
  what follows.  But much has changed since 1958.  In particular,
  philosophers have elaborated detailed accounts of xyz, etc.  In
  addition, Winch seems to have been unaware of Sellars'
  \citetitle{sellars_empiricism_1997} (EPM).
\end{remark}

%%%%%%%%%%%%%%%%
\section{\SR{}: The Received View}

\begin{abstract}
  This section provides a schematic overview of the major features and
  themes of the most common characterizations of \SR{}.  Along the way
  it points out some problems which will be explicated in detail later
  in the paper.  The purpose here is just to present a clear
  exposition of what we're talking about.

  Naturally a brief overview like this risks errors of ommision as
  well as misrepresentation.  But I hope that most readers will find
  it unobjectionable, or at least close enough for the purposes of
  this paper (i.e. clarification and critique.)
\end{abstract}

\subsection{Status of \SR{}}

\begin{remark}
What is ``\SR{}''?  Two answers: research that \textit{uses} surveys
to collect (and construct) data, and research into the nature of
surveys.  First, each survey project studies something, or several
things (each question being ``about'' something).  Second, \SR{} as a
kind of meta-discipline studies surveys; usually this goes by
``\SMeth{}''.

\SR{} counts as a field science, although many survey researchers attempt to conduct ``experiments''; more on this below.
\end{remark}

\begin{itemize}
\item \SR{}: production and collection of social science ``data''
  (more accurately: \textit{facta}, mades, rather than \textit{data},
  givens) by means of questions.  Better: survey \textit{based}
  research; research that \textit{uses} survey techniques to enable
  study of a phenomenon.  How then do we know that these techniques
  are appropriate for the object of study?
\item \SMeth: study of the use of questions to produce and collect data
\end{itemize}

\SR{} is more engineering discipline than science.  Like any
engineering discipline, it \textit{uses} the results of science.  But
unlike genuinely technological engineering disciplines, it cannot rely
solely on scientific knowledge, since it studies human populations.
In this respect it is like product design.  An engineer can design a
functioning cell phone, but the design of a successful phone that
people can easily use requires more than just engineering skill.  That
sort of product design skill is more aesthetic than scientific
(although it too may use science).

There are two aspects here:

\subsubsection{Sampling}

The sampling side of \SR{} is indisputably scientific, in that it uses
mathematical statistics.  However, it also relies on knowledge about
people.

\subsubsection{Interviewing}

A common complaint among \SR{} investigators is that we lack good
scientific models of the survey interview.  So we make do with
whatever findings from psychology, sociology, etc. seem most useful.

\subsection{The \SM{}}

\begin{remark}
  This is a model of survey interviewing, not \SR{}.
\end{remark}

Laboratory model: based entirely on (bad) analogy to the physical
sciences.

\begin{itemize}
\item Telementation (information transfer) model of communication
\item Thought-language dichotomy (Language of Thought model)
\item Cognitivist-mentalist-computational model of thought
\item Stimulus-Response model of question-answer interaction (discursive practice)
\item Performance-Competence model of (skilled) behavior
\item Lab rat model of respondent
\item Lab technician model of interviewer
\item Realist, representationalist model of meaning
\item (or: Information model of meaning)
\item Lab instrument (of measurement) model of questionnaire and questions
\item ``Scientific method'' model of ``interview process''
  (``standardized'' interviewing)
\item Causal model of cognition and action
\item Bilevel, appearance-reality model of phenomena (e.g. ``latent'' variables
  as causes of observable phenomena, etc.)
\end{itemize}

\subsubsection{Cognitivist Model of Thought}

\begin{remark}
  TODO: overview of classic cognitivism and the standard
  Tourangeau-Rips-Razinski model of the ``response process''.
\end{remark}

Two problems with the std cognitivist model.  One is that critics like
Descombes have pretty much destroyed the classic cognitivist model,
and cogsci itself has moved on (e.g. to situated/embedded/extended
models); the other is that pragmatism has offered a radically
different account of mindedness in general, which has no need of the
classic cognitivist model.

\begin{remark}
  Critiques of classic cognitivism: Descombes, Dreyfus, etc. attack on
  conceptual, logical, philosophical grounds; cognitive scientists
  themselves forced to modify the classic model by empirical evidence
  as well as theoretical/philosophical reasoning.  For example, a
  wealth of evidence that the body plays a role in cognition forced a
  reconsideration of the notion that cognition is mental/psychological.
\end{remark}

\begin{remark}
  TODO: classic v. ``2nd wave'' cognitivism
\end{remark}

\subsubsection{Bilevel model}

Evident in frequency of certain sorts of vocabulary in the \SR{}
literature, especially theoretical or methodological literature.

For example:

\begin{itemize}
\item interview \textit{process}
\item response \textit{process}
\item ``tapping'' underlying process etc.
\item ``latent'' variables (required by this model, as causal factors)
\item etc.
\end{itemize}

\begin{remark}
  Such language presupposes and implicitly relies on a causal model of
  behavior.  But the use of such language and its presuppositions is
  virtually never explicitly addressed by survey researchers.  It is
  almost always just accepted without critical examination.
\end{remark}

\subsection{The \XSM{}}

Two kinds of critics of the SM: radical and conservative.

Radical critics argue that the whole thing is rotten and should be
discarded.


Conservative critics tend to admit that the whole thing \textit{may}
be rotten, but they content themselves with trying to improve the
existing model - hence, the \textit{Extended Standard Model} (XSM) of
\SR{}.  The XSM extends the Standard Model by introducing notions of
interaction and conversational dynamics.  But the conservative (or
moderate) camp tends to be quite timid, and seems reluctant to follow
through to the logical conclusion.  They treat \SR{} as something of a
sacred cow.

For example:

\begin{quote}
``One could study the standardized survey interview and conclude that
  it does not work.\ldots{} \textit{This may be a justified
    conclusion.  Conversation analytic studies make visible the weak
    basis of the results of quantitative research.}  Nevertheless, I
  adopt a more pragmatic [\textit{sic}] view and believe that modern
  society cannot and will not do away with this efficient and
  relatively inexpensive measuring instrument.  We should therefore
  try to improve its quality.'' \parencite[Hanneke Houtkoop-Steenstra,
    quoted in][p. 275]{potter_review_2003}
\end{quote}

\begin{remark}
  Here ``pragmatic'' means something like practical, or likely to be
  acceptable by today's researchers; it has nothing to do with the
  philosophical pragmatism that is the topic of this paper.
\end{remark}

\begin{remark}
  Note that H-S is only talking about ``standardized'' interviewing
  technniques, not about the SM in general.  CA does expose problems
  with standardized interviewing (``quantitative research''), but
  pragmatism exposes problems at the more fundamental level of the SM
  itself, of which standardized interviewing is only a component.

  There seems to be little point in trying to trying to improve
  something that ``does not work''.  If it is true (as it no doubt is)
  that ``modern society cannot and will not do away with'' \SR{}, then
  the better course of action is is to reconceptualize the enterprise
  so that it does work.

  The key question, if it does not work, is why.  If it can be shown
  that it \textit{cannot} work, then trying to improve it would
  obviously be pointless.  And on the pragmatist view (explicated
  below), the SM not only cannot work, it is incoherent.
\end{remark}

\begin{remark}
  The structure here, to be explicated in detail below, is analogous
  to the semantics-pragmatics dichotomy: the Extended Standard Model
  only augments the Standard Model.  The alternative, pragmatist
  approach is to expose the irremediable problems with the Standard
  Model, so that it becomes clear that the dichotomy is only apparent.
\end{remark}

\subsection{Survey Methodology}

Monism v. pluralism.

\subsection{Problems}

\begin{itemize}
\item Failure to distinguish space of reasons and space of laws
\item Over-reliance on cognitivism
\item Physics envy
\item Failure to distinguish between philosophical and scientific ``stuff''
\item ...
\end{itemize}

%%%%%%%%%%%%%%%%
\section{Pragmatism}

\epigraph{Consider what effects, that might conceivably have practical bearings, we conceive the object of our conception to have. Then our conception of these effects is the whole of our conception of the object.}
{``The Pragmatic Maxim'' \\ CS Peirce (CP5.402)}

\begin{remark}
  Three ways to proceed: 1. critical: present some strawmen and show
  how pragmatism knocks them down (e.g. representationalism,
  cognitivism, etc.); 2.  constructive: say what pragmatism is,
  without worrying about what it isn't or what it says about competing
  ideas; 3. present some issues or ``problems'' or themes, and discuss
  the various options on offer for addressing them, highlighting the
  pragmatist ones.  The latter is generally Brandom's strategy, and it
  makes for a more irenic, fair presentation.  Knocking down strawmen
  is fun, but runs the risk of aiming at the wrong
  targets.
\end{remark}

\begin{remark}
  ``But on the other hand, `we' all know that when we look and see, when we carefully scrutinize what is bunched together under the rubric of any of these `isms,' we discover assertions, theses, and positions which are conflicting, contradictory, and sometimes incommensurable with each other.  Even those attempts to sort out the varieties of these "isms" are themselves highly contested. So it might seem prudent to drop all `ism' talk and say what we mean without relying on these unstable crutches.  Yet despite all cautionary warnings, we also have a sense that no matter how vague and ambiguous these `ism' expressions are, they can do some important work for us. They enable us to gesture toward philosophic orientations and approaches that do share family resemblances and do have significant consequences.  At the very least, we should avoid falling into the trap of thinking that whenever someone uses one of these contested `ism' terms `we'--more or less--know what is meant.  It is advisable to adopt the maxim that we should always scrutinize what precisely is being said (and unsaid) when someone appeals to any of these `isms.'''  \parencite[57]{baert_towards_2005}
\end{remark}


``Although the term pragmatism is frequently used to characterize some or other highly specific thesis or program, pragmatism is not and never was a school of thought unified around a distinctive doctrine.'' \parencite[1]{talisse_pragmatism_2011}

``As a kind of naturalism, pragmatism is partly a thesis about the
relation of philosophy to the natural sciences; consequently, one
should expect pragmatists to engage the questions of the proper aims
and methods of philosophy.'' talisse 9

``In its most muscular form, the pragmatist thesis is that, once we understand properly the nature of philosophy, we will discover that there are no philosophical problems anyway.'' talisse 9

Pragmatism integrates naturalism (science) and humanism:

``What makes each of these authors pragmatist is their emphasis on
naturalistic and variously humanistic accounts of philosophical
problems and solutions. One of the reasons as to the variety of
pragmatisms is the variety of humanisms available to pragmatists.''
talisse 5

Related: ``the practice turn'' in social theory; interactionism;
dialogism.

\subsection{Historical perspective}

We have the Cartesian, rationalist tradition, and the Romantic,
hermeneutic tradition.  Pragmatism is distinct from both, maybe
because it was so heavily influenced by evolutionary and statistical
thinking.

\subsubsection{Enlightenment and Counter-Enlightenment}

Descartes, Locke, etc.

Vico, Hamann, Herder, etc.

\subsubsection{Analytic and Continental}

In the 20th century (western) philosophy split into the ``analytic''
(mostly Anglophone) and ``continental'' traditions whose practitioners
tended not to talk to each other (much).  Recently this split has been
closing; pragmatism is enjoying a major reinvigoration not only in
Anglophone philosophy departments but throughout Europe.

\begin{remark}
  Pragmatism was distinct from both, and for a period at least treated
  with considerable disdain (both Russell and Heidegger had some
  snotty things to say about it.)  A common narrative has it that
  pragmatism went into eclipse in the mid-20th century, when it lost
  out to logical empiricism, but in fact it never really went away;
  people just didn't use the label.  See the introduction of
  \cite{bacon_pragmatism:_2012} for a good historical overview.
\end{remark}

Examples: Descombes, Brandom, both of whom consciously attempt to
reconcile the two traditions.


Brandom's Geneology of Pragmatism

Brandom traces pragmatism back to Kant.  The lineage runs from Kant
through Hegel, to the American Pragmatists, with branches to the
Heidegger of ?? and the Wittgenstein of
\cite*{wittgenstein_philosophical_2009}, and on to such recent figures
as Quine, Davidson, Putnam, Rorty, etc..

The Continental Tradition

\begin{remark}
    Oddly, Brandom has almost nothing to say about Vico.  Maybe that is because he focuses on the anglophone analytic tradition.
\end{remark}

From Vico to Herder to Foucault, Derrida (etc.)

What about structuralism (Descombes)?

Contemporary pragmatism is conventionally taken to begin with the
great American triumvirate of Peirce, James, and Dewey, but its roots
extend much further back.  Descartes is the great \textit{bete noir}
of most pragmatists, and although Cartesianism came to dominate
philosophical thought it was never unchallenged.  Vico stands out as
an early opponent of Cartesian scientism and defender of humanistic
learning.  For him, science is a matter of geneology; hence his famous
slogan \textit{verum factum}.  We learn what something is by studying
how it came to be.

Vico: ``verum et factum convertuntur, that “the true and the made
are...convertible,” or that “the true is precisely what is made”
(verum esse ipsum factum).'' \parencite{costelloe_giambattista_2012}

\begin{remark}
  Vico, Nietsche, Foucault, etc. - the ``geneological'' tradition.
  This is a history that can be trace as a counterweight to the
  history of cognitivist/empiricist tradition (Descartes, Locke, Hume,
  etc. up through Chomsky, Fodor, etc.)  How much of this do we need
  to cover?  It's mainly of historical interest, but conceptually this
  is of a piece with pragmatism.
\end{remark}

\cite{lollini_becoming_2012}

\cite{egginton_pragmatic_2004}

\subsection{Major Themes}

Negative and positive.

\begin{itemize}
\item Pro:
  \begin{itemize}
  \item normativity
  \item inferentialism
  \item expressivism
  \item semantic externalism
  \item meaning holism
  \item naturalism
  \item language
  \item social cognition (v. atomism)
  \item evolution \& statistics
  \end{itemize}
\item Con:
  \begin{itemize}
  \item causal, nomological basis of action
  \item foundationalism
  \item essentialism
  \item representationalism
  \item cognitivism
  \item mentalism
  \item cognitive/semantic internalism
  \item atomism
  \item methodological individualism
  \end{itemize}
\end{itemize}

``Quine’s corpus presents an ongoing development of a few key
pragmatist and naturalist in- sights about science, language, and
ontology, and an attempt to fit them together. Importantly, Quine
proceeds by way of critical engagement with nonnaturalist critics and
interlocutors....the case for pragmatism was to be made on a
case-by-case basis, not by way of a comprehensive philosophical
system.'' \parencite[8-9]{talisse_pragmatism_2011}

\subsubsection{Practice}
\subsubsection{Normativity}
\subsubsection{Inferentialism}
\subsubsection{Expressivism}

\subsubsection{Deflationism}

Semantic and ontological deflationism, aka minimalism.

\subsubsection{Semantic Externalism}

\subsubsection{Language as Discursive Practice}

\subsubsection{Meaning Holism}
\subsubsection{Primacy of the Social}
\subsubsection{Evolution \& Statistics}

\subsubsection{Naturalism, Mind, Beliefs}

\epigraph{
%% She sang beyond the genius of the sea.
%% The water never formed to mind or voice,
%% Like a body wholly body, fluttering
%% Its empty sleeves; and yet its mimic motion
%% Made constant cry, caused constantly a cry,
%% That was not ours although we understood,
%% Inhuman, of the veritable ocean.
%% \vspace{4pt}
%% The sea was not a mask.  No more was she.
%% The song and water were not medleyed sound
%% Even if what she sang was what she heard.
%% Since what she sang was uttered word by word.
%% It may be that in all her phrases stirred
%% The grinding water and the gasping wind;
%% But it was she and not the sea we heard.
%% \vspace{4pt}
%% For she was the maker of the song she sang.
%% The ever-hooded, tragic-gestured sea
%% Was merely a place by which she walked to sing.
%% Whose spirit is this?  we said, because we knew
%% It was the spirit that we sought and knew
%% That we should ask this often as she sang.
%% \vspace{4pt}
%% If it was only the dark voice of the sea
%% That rose, or even colored by many waves;
%% If it was only the outer voice of sky
%% And cloud, of the sunken coral water-walled,
%% However clear, it would have been deep air,
%% The heaving speech of air, a summer sound
%% Repeated in a summer without end
%% And sound alone.  But it was more than that,
%% More even than her voice, and ours, among
%% The meaningless plungings of water and the wind,
%% Theatrical distances, bronze shadows heaped
%% On high horizons, mountainous atmospheres
%% Of sky and sea.
%% \vspace{4pt}
\hspace{12pt}It was her voice that made\\
The sky acutest at its vanishing.\\
She measured to the hour its solitude.\\
She was the single artificer of the world\\
In which she sang.  And when she sang, the sea,\\
Whatever self it had, became the self\\
That was her song, for she was the maker.  Then we,\\
As we beheld her striding there alone,\\
Knew that there never was a world for her\\
Except the one she sang and, singing, made.\\
%% \vspace{4pt}
%% Ramon Fernandez, tell me, if you know,
%% Why, when the singing ended and we turned
%% Toward the town, tell why the glassy lights,
%% The lights in the fishing boats at anchor there,
%% As night descended, tilting in the air,
%% Mastered the night and portioned out the sea,
%% Fixing emblazoned zones and fiery poles,
%% Arranging, deepening, enchanting night.
%% \vspace{4pt}
\ldots \\
Oh!  Blessed rage for order, pale Ramon,\\
The maker's rage to order words of the sea,\\
Words of the fragrant portals, dimly-starred,\\
And of ourselves and of our origins,\\
In ghostlier demarcations, keener sounds.
}
{\textit{The Idea of Order at Key West} \\
\textsc{Wallace Stevens}}

``The maker's rage to order words of the sea\ldots{} and of ourselves
and of our origins'': the Cartesianist suffers from a delusion: he
thinks that his task is to \textit{discover} order in our doings, an
order that is not our doing.  He yearns to find intentionality in
inner states, to reduce the human to the physical, etc.  The
pragmatist too is seized by ``the maker's rage to order'', but like
the narrator of Stevens' great poem, has no illusions about the source
of the order we perceive.  It is \textit{our} order, an order we
\textit{make}.

\subsection{Second-generation Cognitivism}

\begin{remark}
  Second (and third) generation cognitivism (I take 3rd gen to refer
  to models in which the social plays an essential role) seems to be
  moving clearly in the general direction of some kind of pragmatist
  orientation.  Especially ``enactive'' models.
\end{remark}

What distinguishes pragmatism from 2nd and 3rd generation cognitivism?

Pragmatism - or at least the sort of \textit{rational} pragmatism
elaborated by Brandom - does not sweat the details of the sub- or
pre-conceptual systems studied (postulated) by cognitivists.
Obviously cognition depends on some kind of physical causal basis.
One needs a brain to think.  Some 2nd generation cognitivist models
extend this idea and claim that the sort of thinking of which humans
are capable also requires extra-cranial bodies; for example, some
argue that gesture forms the basis of cognition, and more radical
models propose that extra-bodily phenomena such as pencils and
computers should be counted as components of cognition.  But for the
pragmatist this is beside the point; conceptual activity occurs in the
space of reasons, which is distinct from the space of nature and laws.
So the details of just how we as physical creatures are able to think
are not important or even relevant to the primary question of just
what counts as thinking.  That is a conceptual issue rather than a
scientific one.

The tricky bit for pragmatism is how to account for the move from the
sub-conceptual, law-governed world of (preconceptual) sentience to the
conceptual, rational world of (conceptual) sapience.  There are
several theories on offer for this.  McDowell argues that perception
is always already conceptually structured; Brandom denies this.  Etc.
Either way, the real issue is what to say about concept use in the
space of reasons, regardless of what science may have to say about the
physical basis upon which concept use depends.  And this is a
philosophical rather than a scientific endeavor.  Not that philosophy
can ignore science; pragmatism in general is enthusiastically
naturalistic and happy to defer to science where necessary.  But
science cannot tell the whole story.  In particular, it cannot tell us
what counts as concept use (rationality).

\begin{remark}
This seems too strong.  Price might argue that science can tell us a
great deal about the role that concepts play in our lives.  But you
have to have concepts already before you can undertake that sort of
scientific investigation.
\end{remark}

\begin{remark}
  This is also related the the question of naturalism, another major
  theme in pragmatism.  The space of reasons seems to be distinct from
  the natural space of laws and causes.  So one challenge for
  pragmatism is to provide a naturalistic account of the space of
  reasons.  Price is good on this.
\end{remark}

\subsection{Tasks \& Strategies}

\epigraph{
I wake to sleep, and take my waking slow.\\
I feel my fate in what I cannot fear.\\
I learn by going where I have to go.\\
\vspace{4pt}
We think by feeling. What is there to know?\\
I hear my being dance from ear to ear.\\
I wake to sleep, and take my waking slow.\\
\vspace{4pt}
Of those so close beside me, which are you?\\
God bless the Ground!   I shall walk softly there,\\
And learn by going where I have to go.\\
\vspace{4pt}
Light takes the Tree; but who can tell us how?\\
The lowly worm climbs up a winding stair;\\
I wake to sleep, and take my waking slow.\\
\vspace{4pt}
Great Nature has another thing to do \\
To you and me; so take the lively air,\\
And, lovely, learn by going where to go.\\
\vspace{4pt}
This shaking keeps me steady. I should know. \\
What falls away is always. And is near. \\
I wake to sleep, and take my waking slow. \\
I learn by going where I have to go.}{``The Waking''\\
\textsc{Theodore Roethke}}

``Task'' instead of ``object of study''; ``Strategy'' instead of
``methodology''.  To talk of an object of study would be to
prematurely presuppose that such an object exists and is identifiable
as such.  But pragmatism makes no such presupposition; it learns by
going where it needs to go.

``The task of constructing fully and genuinely post-Cartesian concepts
of concepts and their contents is one we have only begun.'' \parencite[252]{brandom_descombes_2004}

Brandom, Price, etc. adopt similar analytic/explanatory strategies which have
their roots in Peirce's Maxim.

``Roughly speaking, deflationists suggest that semantic vocabulary enables speakers to do useful things with (other, pre-existing) words and sentences - to do things which they couldn't do so well, or at all, without semantic vocabulary...A functional account of this kind is, inter alia, an account of the use speakers make of the semantic vocabulary concerned. It explains the vocabulary in terms of its use and function in the linguistic community. But it does not reduce or analyse facts about meaning to facts about use. Instead it explains talk of meanings, and tells us what it takes to belong to a community who go in for such talk.'' Price, Defl about truth p. 112

``If semantic properties do attach to physical objects in a primary sense, then deflationism is is a non-starter. In particular, it is not enough to try to show that these philosophers are looking for the wrong sort of property - a thick notion of aboutness, where a thin one would do, for example. As deflationists, we need to argue that they looking in the wrong place, that they have the wrong conception of the nature of the problem.'' Price 112

``The solution, I think, is to abandon the idea that among the goals of a use-based theory of meaning should be that of providing a non-semantic reduction of propositions of the form "x means F". On the contrary, I think, the right approach to these locutions is that applied with such success by deflationists in the case of truth: viz. to explain the function of such a locution - in general, the function of talk about meaning - in terms which don't require that it refers to substantial properties.

As noted above, such an approach is bound to appeal to facts about usage. It will tell us under what circumstances speakers use the locutions concerned, and what functions this use serves in the speech communities concerned. But instead of analysing facts about meaning in terms of facts about use, it explains our talk of meanings, and tells us what habits of usage underlie such a discourse.'' same, p. 115

\subsection{Critical aspects}

\begin{remark}
To complement the positive account of pragmatism above, this section
will give an account of pragmatism's criticism of some of the themes
to which it stands in opposition.

Alternatively, we might prefer to integrate this material in the
thematic sections above, so both critical and constructive
perspectives presented together for each theme.
\end{remark}

\subsubsection{Representation}

\epigraph{ [T]he chief service of pragmatism, as regards epistemology,
  will be to give the \textit{coup de grace} to representationalism.}
         {Dewey, letter to Charles Augustus Strong, 1905, quoted in Menand}

Virtually all pragmatists are anti-representationalists

\subsubsection{Mentalism}

In general, pragmatism is anti-mentalistic.  Strong versions deny that
minds exist.  Etc.

\subsubsection{False Dichotomies}

\begin{itemize}
\item quantitative v. qualitative ``variables''
\item Reality-appearance, overt-covert, observable-hidden,
  manifest-latent, etc. ; true-false; etc.; ``hidden'' laws,
  processes, entities ``underlying'' observables, etc.
\item Analytic-Synthetic (Quine)
\item Fact-Value (Putnam)
\item Qualitative-Quanitative
\item Word-World
\item Semantics-Pragmatics
\item Performance-Competence
\end{itemize}

\subsection{Brandom}

Contemporary philosophical pragmatism receives its most complete and
thorough exposition in Robert Brandom's masterpiece \enquote{Making It
  Explicit}\footnote{\MIE{} is over 600 pages of close argument covering most of the philosophical topics of interest to \SR{}.  For a more manageable introduction to Brandom's ideas see his \parencite{brandom_precis_1997}, \parencite{brandom_articulating_2001}.  See also \parencite{brandom_perspectives_2011}, \parencite{brandom_reason_2009}, and \parencite{brandom_between_2008}.}.

\subsubsection{Sellars: Myth of the Given, Space of Reasons}


Natural space of causes (laws), discursive space of reasons

\subsubsection{Sellars: Language Entries}

This is the device that accounts for the relation of causal and
rational orders.  It is true that the world in some sense has a causal
influence on our language performances, but that is not enough to
account for the intelligibility of those performances.  When we
declare ``That's red'' in the presence of red things, we do so
``because'' (in some sense) of those red things and their (causal)
relation to us.  This is what Sellars dubbed a ``language entry''
move.  But that sort of causality cannot account for the conceptual
content of our utterance.

\subsubsection{Brandom: From Sentience to Sapience}

To say ``That's red'' is to apply the \textit{concept} ``red'', and
the subpersonal, causal relation between the presence of a red thing
and our conceptually contentful utterance cannot account for this.  It
cannot account for our ability to apply the concept red
\textit{correctly}, to red things, not non-red things.  After all, if
the presence of red things caused us to say ``That's red'', then we
would in fact say that hundreds or thousands of times a day.  A causal
model cannot account for four fundamental normative aspects of our
behavior: the ability to lie, to err, to hedge (``It \textit{seems}
red''), and to remain silent.

Brandom's Parrot: one of Brandom's favored illustrative examples is a
parrot trained to squawk ``That's red'' in the presence of red things.
This is an example of \textit{sentience} rather than
\textit{sapience}.  Brandom's Parrot is not sapient; its performance
does not count as conceptually contentful (rational), since it does
not involve the application of concepts.  This is where inferential
semantics enter the picture: the content of ``red'' is essentially
inferentially articulated.  To count as a concept user the parrot must
be capable of drawing inferences (either explicitly or implicitly)
involving the concept ``red''.  For example, it must know that
``That's green'' is incompatible with ``That's red''.  Those
inferences, in turn, are only intelligible in terms of what Brandom
(following Sellars) calls ``the game of giving and asking for
reasons''.

Question-based interviews: only intelligible as ``language games'',
denizens of the Space of Reasons.

\begin{remark}
  The fundamental mistake made by the \SM{} is failure to distinguish
  between distinct ``orders of explanation'': the subpersonal, causal
  world, and the personal, discursive, rational world.  Q\&A-based
  interviewing lives in the latter, not the former.  The notion that
  questions are stimuli that ``cause'' responses is fundamentally
  mistaken.  Whatever causal relations may obtain between a question
  utterance and the ensuing response utterance are not relevant to the
  intelligibility of the game.  Responses have \textit{reasons}, not
  causes.
\end{remark}

\begin{remark}
  An example would be useful here.  Maybe ``How old are you?''  A
  correct response to this question is one that involves propositional
  commitments and entitlements.  It does not involve any causal
  relationship to the question, still less to any ``latent'' age
  variable whose value is, say ``27 years''.  Crudely put, you know
  you're 27 years old if you know that last year you were 26.  More
  accurately, you know \textit{how} to respond because you know the
  rules of the language game, which involves also counting years and
  birthdays.  Consider how children learn their ages: they learn that
  certain verbal performances (e.g. ``I'm four'') are correct,
  regardless of whether they understand what they mean, and they learn
  that every year they have a ``birthday'', after which a different
  performance (``I'm five'') is correct.
\end{remark}

\subsection{Vocabularies}

Measurement as description.  Description v. evaluation.  Price on
naturalisms.  The bifurcation thesis.


\subsection{Bibliography}

\noindent
\cite{bacon_pragmatism:_2012} \\
\cite{barnes_ethnomethodology_1985} \\
\cite{baert_pragmatism_2003} \\
\cite{baert_realism_2003} \\
\cite{baert_pragmatism_2004} \\
\cite{baert_towards_2005} \\
\cite{baert_philosophy_2005} \\
\cite{berard_rethinking_2005} \\
\cite{bloor_wittgenstein_2001} \\
\cite{blackburn_invited_1986} \\
\cite{blackburn_steps_2010} \\
\cite{brandom_mie} \\
\cite{brandom_precis_1997} \\
\cite{brandom_articulating_2001} \\
\cite{brandom_pragmatist_2004} \\
\cite{brandom_between_2008} \\
\cite{brandom_reason_2009} \\
\cite{brandom_perspectives_2011} \\
\cite{brandom_analyzing_2011} \\
\cite{brandom_classical_2011} \\
\cite{brandom_vocabularies_2011} \\
\cite{brandom_social_1993} \\
\cite{button_ethnomethodology_1991} \\
\cite{churchill_ethnomethodology_1971} \\
\cite{descombes_minds_2001} \\
\cite{emirbayer_pragmatism_2010} \\
\cite{garfinkel_studies_1984} \\
\cite{garfinkel_ethnomethodologys_2002} \\
\cite{heritage_garfinkel_1984} \\
\cite{kraut_varieties_1990} \\
\cite{loeffler_neo-pragmatist_2009} \\
\cite{lynch_ethnomethodology_2001} \\
\cite{lynch_cognitive_2006} \\
\cite{macdonald_nature_1992} \\
\cite{margolis_reinventing_2002} \\
\cite{margolis_pragmatism_2007} \\
\cite{maynard_diversity_1991} \\
\cite{maynard_toward_2000} \\
\cite{price_true_question_1983} \\
\cite{price_expressivism_2013} \\
\cite{price_naturalism_2013} \\
\cite{price_pluralism_2013} \\
\cite{price_two_2013} \\
\cite{putnam_representation_1991} \\
\cite{putnam_collapse_2002} \\
\cite{putnam_three_2009} \\
\cite{rorty_method_1981} \\
\cite{rorty_representation_1988} \\
\cite{rorty_pmn} \\
\cite{schatzki_practice_2001} \\
\cite{sellars_empiricism_1997} \\
\cite{tate_foucault_2007} \\
\cite{weiss_reading_2009} \\
\cite{winship_ethnomethodology_2010} \\
\cite{zimmerman_review_1994}

%%%%%%%%%%%%%%%%%%%%
\section{Science and Scientism}

\begin{remark}
  Q: What does Pragmatism have to say about science, and why should we
  care?  A: Philosophy as therapy (Rorty) or edification
  (Wittgenstein?).  Exposure of unexamined presuppositions and
  consequences, etc.
\end{remark}

\begin{remark}
  Science as source of \textit{authority} -  epistemic and otherwise. 
\end{remark}

\subsection{The Demarcation Problem}

\begin{abstract}
  This section reviews the demarcation problem: what distinguishes
  science from non- (pseudo-, cargo cult, ...) science?
\end{abstract}

\subsection{Description, Prediction, Action}

\epigraph{%
Last evening the moon rose above this rock \\
Impure upon a world unpurged. \\
The man and his companion stopped \\
To rest before the heroic height. \\
\vspace{4pt}
Coldly the wind fell upon them \\
In many majesties of sound: \\
They that had left the flame-freaked sun \\
To seek a sun of fuller fire. \\
\vspace{4pt}
Instead there was this tufted rock \\
Massively rising high and bare \\
Beyond all trees, the ridges thrown \\
Like giant arms among the clouds. \\
\vspace{4pt}
There was neither voice nor rested image, \\
No chorister, nor priest. There was \\
Only the great height of the rock \\
And the two of them standing still to rest. \\
\vspace{4pt}
There was the cold wind and the sound \\
It made, away from the muck of the land \\
That they had left, heroic sound \\
Joyous and jubilant and sure.%
}{``How to Live. What to Do.''\\
Wallace Stevens}


Rorty's complaint: science cannot tell us what to do, but scientism
thinks it can.

``To sum up this point: there are two distinct desiderata for the vocabulary of the social sciences:
(1) It should contain descriptions which permit prediction and control
(2) It should contain descriptions which help one decide what to do.''
\cite[p. 75]{rorty_method_1981}

\cite{rorty_representation_1988}

\subsection{Naturwissenschaften}

\begin{abstract}
  A concise survey of those features of ``hard'' science that figure
  prominently in the \SR{} literature.  One purpose is to clarify the
  ways in which \SR{} tries to mimic other sciences.
\end{abstract}

\subsubsection{Measurement}
\subsubsection{Replicability}
\subsubsection{etc.}

\subsubsection{Experimental and Field Science}

Radicals may insist that replicable experimentation is essential to
science.  But in astronomy, paleontology, and various other ``field''
sciences, replicable experiments are impossible; does this mean they
are not genuine sciences?  The answer for virtually any reputable
scientist, I hazard to guess, is no: they are in fact scientific
disciplines.

Then there is biology, which involves both experimental and field
research.  Molecular biologists surely conduct genuine experiments;
even evolutionary biologists can investigate evolution by conducting
small-scale experiments.

Then there is the question of simulation.  With the advent of
inexpensive computation, scientists in virtually any field can use
computational simulations to model the real world; does this
non-empirical research count as genuine science?

The point of these considerations is to open up space for a broader
notion of what counts as genuine science.  The idea is that we can
then accomodate the social (or human, etc.) sciences as genuinely
scientific \textit{without} mimicking physics or biology or any other
science.

Or: science as a plurality.  There is no one scientific method that
allows us to distinguish between science and non-science.  Instead,
each science must discover (that is, construct) its own methods.

This means, among other things, that \SR{} need not concern itself
overmuch with interviews as quantitative measurement.  At least not
antecedently; quantitative measurement may be appropriate, in the
right circumstances, but it should not be antecedently imposed as a
criterion of adequacy.  \SR{} interviews can be useful and fully
scientific \textit{even if they do no measuring}.

\begin{remark}
  This is not news; that there is no One True Scientific Method has
  been widely recognized for decades if not centuries.  Many recent
  students of science have explicitly addressed this, most obviously
  Kuhn, Lakatos, Feyerabend, etc.

  So what is the point of bringing this up?  Just clarity, mainly.  In
  the \SR{} literature, as far as I can tell, these issues are rarely
  explicitly addressed.  Many papers on \SR{} ``experiments'' have
  been published, but I have yet to find a detailed examination of why
  we should deem the research described as experimental.  Indeed it
  seems clear that a good deal of this research fits Feynman's
  definition of Cargo Cult Science.  In order to get clear about
  exactly what \SR{} is (and what it is not), how it should (should
  not) be conducted, etc. such considerations are essential and should
  be as explicit as possible.
\end{remark}

\parencite{ryan_replication_2011}

\parencite{hurlbert_pseudoreplication_1984}

\subsection{Geisteswissenschaften: the Two Traditions}

\begin{abstract}
  The two traditions are philosophical; what is the relevance to
  science?  If we were talking about \textit{Naturwissenschaften},
  say, physics, the relevance would be minimal.  But that's not the
  case with the human sciences.  One of the problems with them is that
  they have always had trouble disentangling the scientific from the
  philosophical.  So the purpose of this section is to at least sketch
  the main themes that have played important roles in the human
  (social) sciences.  And the reason that is useful is because it
  helps us place Pragmatism in a conceptual space as well as a
  historical context.  Which in turn will help us better to grasp the
  significance of Pragmatism for \SR{}.
\end{abstract}

\begin{remark}
  Winch - tracing social science to Mill, etc. - the analytic branch?
\end{remark}

\begin{remark}
  Sociology v. anthropology?  Different lineages?
\end{remark}

\begin{remark}
  ``Social physics'' - Comte?
\end{remark}

\subsubsection{The Analytic Tradition}

\begin{remark}
  Does this belong under Geisteswissenschaften?  Yes, insofar as it is
  a philosophical program.  But empiricism is obviously ``about'' the
  natural sciences.
\end{remark}

French Rationalism $+$ English Empiricism $=$ Logical Empicism

\subsubsection{The Hermeneutic Tradition}

``Continental'' - Herder, Hegel, ... Foucault, Derrida, ...

\cite{forster_herder_2010}

\subsection{Scientism}

\begin{remark}
  Putnam: ``science-worship''
\end{remark}

\begin{remark}
  Scientism as imperialistic Naturwissenschaften?
\end{remark}

\begin{remark}
  The major point: the Pragmatic Enlightenment dethroned
  ``objectivity'' or ``Reality'' as the source of unimpeachable
  external authority over our intellectual lives, just as the 17th
  century Enlightenment remove religion as the source of authority
  over our political and civic lives.  It follows that pragmatism
  undermines science's self image as the one true objective external
  source of authoritative knowledge.  For pragmatism, science is one
  of many human vocabularies, with no legitimate claim to special
  authority as an external, independent arbiter of truth claims.
\end{remark}

\begin{remark}
  This has special significance for the human sciences.  Among other
  things, it suggests social scientists should stop trying to mimic
  physics.  Not for the traditional reason (i.e. that this is
  impossible), but because physics should not be granted such special
  status.  Why should we take physics as the model science?
  Undoubtedly because it is so successful in practice, as predicting
  and manipulating the world.  But practical success is not the same
  as epistemic authority.

  In fact it seems we are seeing a shift in the relative prestige of
  scientific fields; these days biology seems to be displacing physics
  as the model science.
\end{remark}

\cite{pinker_science_2013}

\cite{wieseltier_crimes_2013}

\cite{pinker_why_2004}

\cite{rorty_philosophy-envy_2004}

\subsubsection{Physics Envy}

  For a long time physics was the model science; it is now being
  displaced by biology as the new model science.  But why should the
  human sciences try to mimic other sciences?  Why not treat them as
  legitimage sciences in their own right?

  Part of the problem is the notion that quantitative measurement is
  the gold standard.  No quantitative measurement, no science.  But
  why did people come to think that?  Probably because of the dazzling
  success of quantitative sciences like physics.  But this is a false
  idol; there is nothing intrinsic to quantitative measurement or the
  ``scientific method'' that that can tell us that quantitative
  measurement is essential to science.  In other words, science cannot
  set its own demarcational criteria; they must come from outside of
  science.  Just because physics is grounded in quantitative
  measurement does not mean that quantitative measurement demarcates
  science.  You can do perfectly good science without measurement.
  Should we not count the anthropologists analysis and description of
  myth and ritual as science?  In the end it is about the production
  of knowledge, by whatever means.

  Science as critical, rational thinking (analysis, but also
  imagination) v. replicable experiments v. field science.  Consider
  the example of philology, which was the model for science in the
  19th century.  Philology is not about discovering laws of nature (or
  of man or society), but about acquiring knowledge of languages and
  texts.  It is (was?) highly disciplined, rational, critical, etc. -
  it has all the characteristics of scientific thinking, but it does
  not have replicable experiments.  It involves very detailed
  collection and analysis of data, construction and ``testing'' of
  hypotheses, is open to revision, and so forth - all the things that
  science has, except for laboratory experiments.

  Ditto for the ``field sciences'' like paleontology, astronomy, etc.

  The point of reviewing the ways in which different research topics
  and programs count as science is to buttress the suggestion that the
  human sciences generally and social sciences in particular should
  spend a lot less energy trying to mimic the ``hard'' sciences.
  Instead they should concentrate on what can make them scientific in
  their own right, on answering the normative question of what counts
  as good science in the study of things human.

  This inevitably involves philosophical questions of the sort that
  help us draw the distinction between philosophy and science.

\subsection{Pragmatism and Science}

\begin{abstract}
  Pragmatism has been intimately connected to science from the very
  beginning; esp. evolution and statistics. (see Brandom) It has to
  potential to unify Geistes- and Naturwissenschaften. (see Price on
  global expressivism, pragmatism about causation, etc.)
\end{abstract}

From Empiricism to Pragmatism

Sellars, Quine, Wittgenstein and the demolition of Empiricism.

Empiricism smuggles the conceptual into perception.  Pragmatism
remedies this by first recognizing that ``all perceptual awareness is
conceptual'' (Sellars, somewhere), and second, that the conceptual is
fundamentally pragmatic (and inferential).

With respect to measurement, this means that the appeal to
isomorphisms between mathematical and ``empirical structures'' is
problematic.  Empirical structures are already conceptual.  They are
not ``given''.  Pragmatism shifts the focus to the practices in virtue
of which we are able to recognize the empirical as such in the first
place.  So the isomorphism of measurement theory must be supplemented
by an account of the relation of the conceptual structure of empirical
systems to the world.  And this relation is essentially pragmatic, a
matter of what we do, how we interact with our external environment,
rather than an antecedently established correspondence between our
concepts and the world.

This can be illustrated in the history of temperature measurement,
where theory and its relation to measurement practice played the
decisive role.

\begin{remark}
  For the moment this is organized by figure.  It would probably be
  better to organize it by theme.
\end{remark}

\subsubsection{Peirce, James, Dewey}

\subsubsection{Quine}

\subsubsection{Putnam, Davidson?}

\subsubsection{Rorty}

Science as one among many vocabs, with no special claim to authority.

\begin{remark}
  Rorty on ``value-free'' and ``hermeneutic'' social science. \cite{rorty_method_1981}
\end{remark}

\subsubsection{Price}

\begin{remark}
Price's global expressivism: all vocabularies, including empirical
vocabularies, are expressive rather than descriptive.  Significance of
this for \SR{}, esp. measurement.
\end{remark}

\begin{remark}
  Price's pragmatism about causality: causes are not in nature.  Our
  cause talk reflects our position in the world and our need to
  intervene and cope.  Significance for \SR{}, esp. for the causal
  models common to most orthodox \SR{}.
\end{remark}

%%%%%%%%%%%%%%%%%%%%
\subsubsection{Deflationism}

\begin{abstract}
Semantic and metaphysical deflationism works as well for validity as
it does for truth and reference.
\end{abstract}

\begin{remark}
  Deflationism seems to depend essentially on some form of
  expressivism.  Or maybe they amount to the same thing?
\end{remark}

\begin{itemize}
\item Deflationism about truth
\item Deflationism about validity
\item Deflationism about quantities
\end{itemize}


%%%%%%%%%%%%%%%%%%%%%%%%
\subsubsection{Causality and the Space of Reasons}

\begin{abstract}
abstract
\end{abstract}

\noindent
\cite{abell_narrative_2004} \\
\cite{crane_mental_1995} \\
\cite{gross_pragmatist_2009} \\
\cite{jackson_mental_1996} \\
\cite{lowe_causal_1993} \\
\cite{lowe_non-cartesian_2006} \\
\cite{macdonald_mental_1986} \\
\cite{menzies_causation_1993} \\
\cite{morris_causes_1986} \\
\cite{williamson_broadness_1998}

\subsubsection{Hypothetical Entities}

\subsubsection{Personal v. Subpersonal}

%%%%%%%%%%%%%%%%
\section{Reconciliation: A Pragmatic Model of Survey Research}

\subsection{The Deontic Scorekeeping Model of Discursive Practice and \SR{}}

\begin{abstract}
Why the deontic scorekeeping model is preferable to others, esp. the
cognitive model.
\end{abstract}

\begin{remark}
  It's a model of discursive, that is rational, practice.  Contrast
  this with most models on offer which tend to focus on subpersonal
  processes; hence the prevalence of talk about ``the survey
  process'', the ``response process'', etc.
\end{remark}

\subsection{A Quality Assurance Model for \SR{}}

\begin{abstract}
abstract
\end{abstract}

%%%%%%%%%%%%%%%%
\section{Notes}

\subsection{Evolution}

Instead of "the QA process", the proper object of investigation is the
local evolution of discourse.

EM studies local produced order.  It may come up with a structural
description.  But locally produced order is the outcome of an
essentially evolutionary process - the mutual adaptation of the
participants to each other and the context.  Also, any such model may
not (probably will not) generalize.  But what does generalize is the
evolutionary mechanism itself, just like in biology.

Rational selection as the mechanism of the evolution of discursive
performances.  What accounts for the deontic attitudes we adopt
regarding performances?  Brandom's account describes the architecture
of such posturings and the significances the institute.  But it does
not really address the logic of discourse as an evolutionary process.

The idea is that Brandom provides an account of discourse qua rational
action.  Different attitudes are endorsed or undertaken for reasons -
that is the source or ground of the intelligibility of discursive
practice.  So if we view the unfolding of discourse as being governed
by the logic of evolution, we can treat Brandom's sort of rational
pragmatism as the selection mechanism that accounts for why some
attitudes (meanings) survive (are endorsed) and others do not.
Meanings that survive must fit into the space of reasons - they must
be assertable and justifiable, even if the participants are unable to
explicitly articulate this.  This makes the evolution of discourse
intelligible as a rational process, rather than a natural process.
Responses to questions are not explicable as effects caused by "true
values" or the like; this would make them fundamentally non-rational.
Or to borrow a bon mot from Garfinkel, this would make respondents
"rational dopes".

Similar language: "negotiation", e.g. "...I suggest that the content
of talk indicates that imposed hierarchies are continually
re-negotiated..."  Negotiation as rational evolution?

The "true score" and other orthodox models account for sentience, not
sapience.

\subsection{Verum Factum}

Cartesianism (spectator, etc.) inspection, discovery, certainty,
foundationism (external foundation grounding knowledge) v.

Verum Factum, geneological/historical, following growth/development,
not certainty but ???; no foundationism, no priviledged vocab, no
external source of authority

Critical notions: authority.  For evidence etc. key idea is authority - the only
kind of authority is the kind we assent to.  So the question is what
do we treat as authoritative and why, rather than how can we discover
the One True external foundational source of authority and learn to
speak its language

Critical notions: vocabulary.  Regardless of what there is, we can
only talk about it by using vocabs.

Relevance to SR: we make our truths, by engaging in dialog with
respondents in order to teach/train them to understand what we want.
In other words we work to make our scorecards converge.  We can never
be sure that researchers and respondents understand each other, have
the same interpretations of qx text, etc.  But we can do what nature
does in evolution and learning: institute a cyclic process of
experiment, feedback, and correction.  This is operational even at the
most simple and basic level of communication.  So we can use this fact
to our advantage.

Communication interactions as not essentially different from processes
of evolution and learning.  Evolutionary process tend to coordinate
organism and environment; learning processes adapt the learner to the
task environment, etc.  Any discursive exchange - even simple
greetings, etc. - does the same sort of thing: coordinate and mutually
adjust the parties to the exchange.

\subsection{Rational Evidence}

Evidence-Based Rational SR

RCT: isolate the causal factor that links Treatment to Outcome

THe mistake make by orthodox SR (shown by its vocab of measurement,
error, etc.) is that it confuses the space of causes and the space of
reasons.

In RCT, we observe a stimulus followed by a response (T followed by O)
and postulate a causal relation.  In SR, we observe a Q performance
followed by a R performance.  In fact this is an idealization since Q
and R cannot be isolated - they are both joint performances.  Ignore
that for now; the point is that what makes them intelligible as
performances is the space of reasons, not causes.  That is, as
discursive episodes they are essentially rational in a way the T-O
trials are not.  By definition, "rational" means involving concepts.
Stimulus-response does not involve concepts and so is not rational in
this favored sense.  The natural world may be lawful, but it is not
rational.

So SR should abandon the orthodox vocab of measurment, etc. in favor
of one involving rationality.  What would "evidence-based" mean, then?
Not the kind of evidence involve in natural science, since such
evidence does not involve concepts and thus meaning.  Instead evidence
inescapably involves meaning and understanding.  What counts as
evidence is what we count as a rational explanation or story.  And
this necessarily involves the perspective of the participants - it is
their rationality, their giving and asking for reasons, that provides
the observational basis of evidence.

 One consequence: Qx does not involve measurement.  SR can use stats
 to statistically measure the collected data, but that is quite
 separate from whether the data measure anything.  So you can say that
 x\% of resondents pick option X, but that does not mean that you have
 measured the distribution of "true values" of some latent variable.
 What you have measure is a distribution of deontic scores, or
 discursive postures.  There is no warrant for claiming that each
 member of the x\% means the same thing by picking X.

\subsection{Misc}

1.  What is a question?  Better: what counts as a question, what is it to ask a question?

2.  Ditto for answer.

Q and A as parts of a whole (holistic view)

Q token v. Q performance, etc.

\subsection{Erotetic Discursive Practice}


EDP as production of data rather than discovery of truth

\subsection{Replication}

Goal is replication.  Compare: blood work, e.g. measuring
cholesteral.  The measuring apparatus reacts to the sample, not the
other way around.  For EDP, respondent reacts to the question, so the
question is analogous to the blood sample.  The response is a kind of
measurement of the question, not the other way around.

Replicability means same setup, same experimental conditions; in EDP
this means replication of conceptual structure, which is accomplished
by the dialog preceding the question.  Traditionally, "ask the same
question"; in practice this is impossible, since what counts is not
the question text but respondent's grasp of the sense.  So the
"experimental setup" should be viewed as the work of teaching the
respondent what the sense of the question is.  Survey interviewing is
essentially interventionist, but this is not necessarily a bad thing,
since lab experiments are too - they "intervene" to set up
experimental "initial conditions".  The difference is that setting up
initial conditions ("same meaning") in question asking means tutoring
the respondent.

\subsection{Myths and Mythologies}

\begin{itemize}
\item The Myth of Question Independence says that the meaning of a
    question is independent of context.  But the meaning of a question
    is always dependent on what came before it.
\item Myth of Autonomy. Interviewer and Respondent.
\item Myth of Error
\end{itemize}

\subsection{Dopes}

Garfinkel's dopes - cultural, judgmental, psychological

Dehumanization.  Orthodox Survey Research (OSR) dehumanizes
participants.  The R is a sampling unit.  The mythology of OSR
measurement treats the human R as a natural object to be measured
rather than a person.


\clearpage
%%%%%%%%%%%%%%%%%%%%%%%%%%%%%%%%%%%%%%%%%%%%%%%%%%%%%%%%%%%%%%%%
\appendix\begin{appendices}

\section{Recommended Reading}

For readers unfamiliar with contemporary pragmatism:

\begin{enumerate}
\item \cite{bacon_pragmatism:_2012}
\item \cite{brandom_perspectives_intro_2011}
\item \cite{putnam_three_2009}
\item \cite{quine_two_1951}
\item \cite{sellars_empiricism_1997}
\item \cite{davidson_nice_2005}
\item \cite{putnam_collapse_2002}
\item \cite{rorty_pmn}
\item \cite{brandom_articulating_2001}
\item \cite{price_expressivism_2011}
\item \cite{price_truth_2003}
\end{enumerate}


\subsection{Pragmatism - Overviews}
\begin{description}
\item [\cite{bacon_pragmatism:_2012}]  A good overview of contemporary pragmatism, covering Quine, Putnam, etc.  Probably the best place to start for those unfamiliar with contemporary pragmatism.
\item [\cite{bernstein_pragmatic_2010}]
\item [\cite{brandom_perspectives_2011}]
\item [\cite{dickstein_revival_1998}]
\item [\cite{kraut_varieties_1990}]
\item [\cite{putnam_pragmatism_1995}]  Addresses what pragmatism has to say about verificationism.
\item [\cite{schatzki_practice_2001}]
\end{description}

\subsection{Pragmatism Readers}

Many collections of writings from the Pragmatist tradition have been
published.  Here are just a few:

\begin{description}
\item [\cite{menand_pragmatism_1997}]
\item [\cite{haack_pragmatism_2006}]
\item [\cite{misak_new_2007}]
\item [\cite{talisse_pragmatism_2011}]
\end{description}

\subsection{(Anti-) Cognitivism}

\epigraph{[C]ognitivism, as an account of human thought and understanding, is deeply false.}{\cite{haugeland_closing_2004}}

\subsubsection{Classic Cognitivism}

``Classic'' or first generation cognitivism is characterized by three
ideas: computation, representation, and the mental.  Roughly,
cognition is construed as mental computational processing of
representations (symbols).

\begin{description}
\item [\cite{descombes_minds_2001}] A major and devastating critique
  of classic cognitivism.  Descombes' positive account of
  ``mindedness'', intentionality, etc. dovetails quite nicely with
  Brandom's; see \cite{brandom_descombes_2004} and
  \cite{descombes_replies_2004}
\item [\cite{brandom_cog_2009}]
\item [\cite{brandom_descombes_2004}]  Comment on \cite{descombes_minds_2001}
\item [\cite{fodor_language_1979}]  The classic philosophical account of cognitivism.
\item [\cite{fodor_lot_2010}]
\item [\cite{haugeland_closing_2004}]  Comment on \cite{descombes_minds_2001}
\item [\cite{rorty_brain_2004}]  Comment on Descombes
\item [\cite{taylor_descombes_2004}]  Comment on Descombes
\item [\cite{descombes_replies_2004}]  Replies to comments
\item [\cite{kitzinger_after_2006}]
\end{description}

\subsubsection{Second-generation Cognitivism}

Common labels: embodied, embedded, extended, enacted, distributed,
situated cognition; the extended mind, the social brain, etc.:

\begin{description}
\item [\cite{adams_embodied_2010}]
\item [\cite{cash_cognition_2013}]
\item [\cite{clark_being_1997}]
\item [\cite{clark_extended_1998}]  The original ``extended mind'' article.
\item [\cite{dunbar_social_1998}]
\item [\cite{ignatow_theories_2007}]
\item [\cite{kono_extended_2010}]
\item [\cite{leidlmair_after_2009}]  A collection of papers dealing with the reassessment of thinking in Cognitive Science and in Philosophy today, with special emphasis on embodied and embedded cognition.
\item [\cite{varela_embodied_1992}]
\item [\cite{walter_situated_2013}]  A very readable overview of first and second generation cognitivism.
\item [\cite{wilson_six_2002}]
\end{description}

\subsection{Pragmatism and the Social Sciences}
\begin{description}
\item [\cite{baert_towards_2005}]
\item [\cite{baert_philosophy_2005}]
\item [\cite{emirbayer_pragmatism_2010}]
\item [\cite{kivinen_relevance_2004}]
\item [\cite{kivinen_sociologizing_2007}]
\item [\cite{wolfe_pragmatic_revival_1998}]
\end{description}

\subsection{Major Figures}
\subsubsection{Robert Brandom}
\begin{description}
\item [\cite*{brandom_mie}]  His \textit{magnum opus}.
\item [\cite*{brandom_articulating_2001}] A much shorter summary of the key ideas in MIE.
\item [\cite*{brandom_perspectives_2011}]
\item [\cite*{brandom_classical_2011}]
\item [\cite*{brandom_analyzing_2011}]
\end{description}

\subsubsection{Donald Davidson}
\begin{description}
\item [\cite*{davidson_nice_2005}] ``There is no such a thing as a
  language, not if a language is anything like what many philosophers
  and linguists have supposed.''
\end{description}

\subsubsection{Huw Price}

Price started out as a philosopher of science.  In recent years he has
focused on the problems of naturalism, expressivism, etc.  He is
associated with:

\begin{itemize}
\item Global expressivism
\item Subject naturalism
\item Causal perspectivism
\end{itemize}

\begin{description}
\item [\cite*{price_expressivism_2013}]
\item [\cite*{price_naturalism_2013}]
\item [\cite*{price_pluralism_2013}]
\end{description}

\subsubsection{Hilary Putnam}
\begin{description}
\item [\cite*{putnam_meaning_1975}] Contains Putnam's famous
  ``twin-earth'' thought experiment, and his famous conclusion
  ``meanings just ain't in the head''.
\item [\cite*{putnam_pragmatism_1995}]
\end{description}

\subsubsection{Willard Van Orman Quine}
\begin{description}
\item [\cite*{quine_two_1951}]  Quine's celebrated demolition of the distinction between analytic and synthetic sentences.  A must-read.
\item [\cite*{quine_word_1960}] Chapter 2, \textit{Translation and
  Meaning}, introduces the famous \textit{gavagai}.
\end{description}

\subsubsection{Richard Rorty}
\begin{description}
\item [\cite*{rorty_pmn}]
\item [\cite*{rorty_method_1981}]
\item [\cite*{rorty_representation_1988}]
\end{description}


\subsubsection{Wilfrid Sellars}
\begin{description}
\item [\cite*{sellars_empiricism_1997}]  Sellars' celebrated attack on the ``Myth of the Given''.  A must-read.
\end{description}

\subsubsection{Ludwig Wittgenstein}
\begin{description}
\item [\cite*{wittgenstein_philosophical_2009}]
\end{description}

%%%%%%%%%%%%%%%%
\section{Bibliography}
%% \addcontentsline{toc}{chapter}{Bibliography}
%% \bibliographystyle{plainnat}
\printbibliography[heading=none]
\end{appendices}

\end{document}
