\documentclass[11pt,twoside]{article}
\usepackage[toc,page,header]{appendix}
\usepackage{pdfpages}
\usepackage{csquotes}
\usepackage{epigraph}
\usepackage{changepage}
\usepackage{fontspec}
\defaultfontfeatures{Scale=MatchLowercase}
\setmainfont[Mapping=tex-text]{Times New Roman}
\setsansfont[Mapping=tex-text]{Arial}
\setmonofont{Courier}

\usepackage{float}
\usepackage{turnstile}
\usepackage{bussproofs}

\usepackage{geometry}
\geometry{letterpaper}

\newtheorem{theorem}{Theorem}
%\newtheorem{cor}{Corollary}
%\newtheorem{lem}{Lemma}
%\theoremstyle{remark}
\newtheorem{remark}{Remark}

\newtheorem{objection}{Objection}
\newenvironment*{response}[1][]{\noindent
\textbf{Response to Objection #1.}
\begin{adjustwidth}{1em}{1em}
}
{\end{adjustwidth}
\vspace{1ex}
}


%\usepackage[parfill]{parskip}    % Activate to begin paragraphs with an empty line rather than an indent

\usepackage{graphicx}
\usepackage[leftcaption]{sidecap}
\sidecaptionvpos{figure}{c}

%\usepackage{amssymb}

\usepackage{epstopdf}
\DeclareGraphicsRule{.tif}{png}{.png}{`convert #1 `dirname #1`/`basename #1 .tif`.png}

\usepackage[
bibstyle=numeric,
citestyle=authortitle,
natbib=true,
hyperref,bibencoding=utf8,backref=true,backend=biber]{biblatex}

\usepackage{hyperref}
\hypersetup{
    bookmarks=true,         % show bookmarks bar?
    unicode=true,          % non-Latin characters in Acrobat’s bookmarks
    pdftoolbar=true,        % show Acrobat’s toolbar?
    pdfmenubar=true,        % show Acrobat’s menu?
    pdffitwindow=false,     % window fit to page when opened
    pdfstartview={FitH},    % fits the width of the page to the window
    pdftitle={Pragmatism and Survey Research},    % title
    pdfauthor={Author},     % author
    pdfsubject={Subject},   % subject of the document
    pdfcreator={Creator},   % creator of the document
    pdfproducer={Producer}, % producer of the document
    pdfkeywords={keyword1} {key2} {key3}, % list of keywords
    pdfnewwindow=true,      % links in new window
    colorlinks=true,       % false: boxed links; true: colored links
    linkcolor=blue,          % color of internal links
    citecolor=blue,        % color of links to bibliography
    filecolor=magenta,      % color of file links
    urlcolor=cyan           % color of external links
}
\usepackage{draftwatermark}


\usepackage{fancyhdr}
\setlength{\headheight}{15.2pt}
\pagestyle{fancy}

\lhead[Pragmatism \& Survey Research]{\thepage}
\chead[]{}
\rhead[\thepage]{Pragmatism \& Survey Research}

\title{Pragmatism and Survey Research}
\author{G. A. Reynolds}
\date{\today}
\bibliography{%
../bib/abstracts.bib,%
../bib/biology.bib,%
../bib/causality.bib,%
../bib/em.bib,%
../bib/logic.bib,%
../bib/mind.bib,%
../bib/philosophy.bib,%
../bib/pragmatism.bib,%
../bib/psychomet.bib%
../bib/psychometrics.bib,%
../bib/misc.bib,%
../bib/measurement.bib,%
../bib/psychology.bib,%
../bib/variables.bib,%
../bib/val.bib,%
../bib/validity.bib,%
}

%% Macros

\newcommand{\SM}{Standard Model}
\newcommand{\XSM}{Extended Standard Model}

\newcommand{\SMeth}{Survey Methodology}

\newcommand{\SR}{Survey Research}
\newcommand{\sr}{survey research}
\newcommand{\SRIV}{Survey Interview}
\newcommand{\sriv}{survey interview}
\newcommand{\SIV}{Survey Interviewing}
\newcommand{\FI}{Field Interviewer}
\newcommand{\Iver}{Interviewer}
\newcommand{\R}{Respondent}
\newcommand{\LPR}{Legal Permanent Resident}
\newcommand{\ART}{Assimilated Response Technique}
\newcommand{\GAM}{Grouped Answer Method}
\newcommand{\IOM}{Instrument of Measurement}

\includeonly{%
%% pilots,cards
}
%%%%%%%%%%%%%%%%%%%%%%%%%%%%%%%%%%%%%%%%%%%%%%%%%%%%%%%%%%%%%%%%
\begin{document}
\maketitle
\nocite{*}

\begin{abstract}
abstract
\end{abstract}

\tableofcontents
\listoffigures

\newpage
%%%%%%%%%%%%%%%%%%%%
\section{Introduction}

\begin{abstract}
Generally speaking, \sr{} is dominated by what might be called
scientistic cognitivism.  The centrality of cognitivism is clearly
evident in the dominant models of ``the survey process'' and the
practices of ``cognitive interviewing''.  The ``scientistic'' part is
evident in the sort of language that dominates survey methodology,
which routinely treats questionnaires as instruments of measurement
and models interviewing on the experimental methods of the physical
sciences.

Meanwhile a quiet revolution has been underway for the past several
decades in the human sciences (including e.g. AI and neuroscience).

The ``new sciences'' - ``cognitive'' this or that - are generally
speaking neither new nor particularly scientific.  The only genuinely
new element is computation.  The emergence of a well-defined concept
of computability in the first half of the 20th century did indeed mark
a conceptual innovation of truly historic proportions.  But the
various ``cognitive'' sciences to which it gave rise, once scholars
began to take a computational perspective on psychology, were not
revolutionary; they only advanced an agenda that has its roots in the
17th century Enlightenment.  Those ``cognitive'' sciences in which
computation plays a central role seek to mathematicize the human, just
as Galilleo, Newton, and other Enlightenment scientists have sought to
mathematicize nature.

The truly revolutionary movement is marked by the development of
Pragmatism.  Properly understood, Pragmatism doesn't advance the
agenda of the first Enlightenment; it turns it upside down.

Pragmatism liberates us from the tyranny of objective reality (Truth,
etc.) without stranding us in a jejune relativism.  It cheerfully
accepts the existence of the real world and the constraints it imposes
on us, but it rejects the notion that we can somehow find the
Archimedean point of purchase that will allow us to prize apart the
real from the apparent, the True from the False.  It denies that there
is any one true method, scientific or otherwise, that will lead us to
the promised land of True Knowledge.  It denies that we can learn to
speak the one true language of nature, or that such a language even
exists.  
\begin{remark}
But it also denies, for all that, that we are condemned to ignorance
and error.  It insists that we can learn, that we can cope with each
other and our environments in the ways that matter.  Etc.  TODO: state
the positive case in a way that ties it back to our ordinary
intuitions of truth, objectivity, etc.
\end{remark}

Up to now, however, the Pragmatist Enlightenment has had relatively
little impact on \sr{}.  Even the relatively small number of sr{}
researchers who have tried (since at least the early 90s) to draw
attention to the contextual and interactive aspects of survey
interviewing have tended to accept the main commitments of the
traditional cognitivist perspective.  They tend to treat context and
interactivity as important but essentially peripheral aspects of a
``process'' whose center remains firmly entangled by cognitivist and
representationalist commitments: to mental entities and processes, the
autonomy of language, the atomicity of words, representational
semantics, and so forth.

The \sr{} literature shows distinct signs of a cargo-cult science
mentality.  A clear example is the use of the term ``probe'' in
discussions of cognitive interviewing.  The metaphor is obvious: a
probe is a scientific instrument used to examine a specimen.  So long
as this is treated as nothing more than a metaphor there is no
problem; but the ``theory'' of cognitive interviewing tends to take it
much farther.  It takes the notion of a probe literally, and construes
ordinary questions as scientific instruments designed to probe the
cognitive architecture of responding subjects.  The clear implication
is that there is something distinctive about the ``probes'' used in
cognitive interviewing, something that makes them scientific
instruments, when in fact they are nothing more than ordinary
discursive performances.  Merely calling a follow-up question a
``probe'' does make it an instrument; still less does it make it
``scientific''.  To pretend otherwise is to engage in cargo-cult
science.  To put it another way: the cognitivist theory under which we
are to treat some verbal performances as ``probes'' is the
\textit{only} justification we have for thinking they are scientific
instruments.  But it provides no means of distinguishing a set of such
performances from any other discursive performance, no way of deciding
what counts as a probe, other than its own theoretical claims.  The
logic is entirely circular.

Critical v. constructive

The purpose of this paper is two-fold.  First it provides an overview
of contemporary Pragmatism, in order to give substance to the claim
advanced above as to its revolutionary character.  One way to do this
is by contrasting it with its opposite number(s), so the result will
be to expose and clarify the fundamental themes and commitments of
representationalism and cognitivism, etc.  This will sharpen some
boundaries.

Once the background issues - theoretical, methodological, philosophical
- are clear, the other task is to examine, at least in a preliminary
manner, the implications of Pragmatism for \SR{}.  What is on offer is
a radical re-conceptualization of the entire enterprise.  A move away
from scientism toward a more properly (and appropriately)
anthropological perspective.

Caveat: we are not talking here about merely methodological issues.
It is not a question of doing the same thing, only better; of finding
an innovative method that solves the old problems.  What is suggested
is rather a fundamental change in the way we conceive of the task,
asking different questions, discarding the old questions as not useful
or even very meaningful.

\end{abstract}

\section{\SR{}}

Field v. experimental science \parencite{ryan_replication_2011},
\parencite{hurlbert_pseudoreplication_1984}


\subsection{\SR{} Models}

\begin{remark}
What is ``\SR{}''?  Two answers: research that \textit{uses} surveys
to collect (and construct) data, and research into the nature of
surveys.  First, each survey project studies something, or several
things (each question being ``about'' something).  Second, \SR{} as a
kind of meta-discipline studies surveys; usually this goes by
``\SMeth{}''.
\end{remark}

\begin{itemize}
\item \SR{}: production and collection of social science ``data''
  (more accurately: \textit{facta}, mades, rather than \textit{data},
  givens) by means of questions.  Better: survey \textit{based}
  research; research that \textit{uses} survey techniques to enable
  study of a phenomenon.  How then do we know that these techniques
  are appropriate for the object of study?
\item \SMeth: study of the use of questions to produce and collect data
\end{itemize}

\subsubsection{\SM{}}

Laboratory model: based entirely on (bad) analogy to the physical
sciences.

\subsubsection{\XSM{}}

XSM = SM plus interaction

\subsection{Survey Methodology}

Monism v. pluralism.

\section{Pragmatism}

\epigraph{Consider what effects, that might conceivably have practical bearings, we conceive the object of our conception to have. Then our conception of these effects is the whole of our conception of the object.}
{CS Peirce (CP5.402)}

``Although the term pragmatism is frequently used to characterize some or other highly specific thesis or program, pragmatism is not and never was a school of thought unified around a distinctive doctrine.'' talisse 1

``As a kind of naturalism, pragmatism is partly a thesis about the
relation of philosophy to the natural sciences; consequently, one
should expect pragmatists to engage the questions of the proper aims
and methods of philosophy.'' talisse 9

``In its most muscular form, the pragmatist thesis is that, once we understand properly the nature of philosophy, we will discover that there are no philosophical problems anyway.'' talisse 9

Pragmatism integrates naturalism (science) and humanism:

``What makes each of these authors pragmatist is their emphasis on
naturalistic and variously humanistic accounts of philosophical
problems and solutions. One of the reasons as to the variety of
pragmatisms is the variety of humanisms available to pragmatists.''
talisse 5
\subsection{Major Themes}

Negative and positive.

\begin{itemize}
\item anti-foundationalism
\item anti-representationalism
\item normativity
\item inferentialism
\item expressivism
\item cognitivism, mentalism
\item naturalism
\item evolution \& statistics
\end{itemize}

``Quine’s corpus presents an ongoing development of a few key
pragmatist and naturalist in- sights about science, language, and
ontology, and an attempt to fit them together. Importantly, Quine
proceeds by way of critical engagement with nonnaturalist critics and
interlocutors....the case for pragmatism was to be made on a
case-by-case basis, not by way of a comprehensive philosophical
system.'' Talisse intro p. 8-9

\subsubsection{Sellars: Myth of the Given, Space of Reasons}

\subsubsection{Sellars: Language Entries}

This is the device that accounts for the relation of causal and
rational orders.  It is true that the world in some sense has a causal
influence on our language performances, but that is not enough to
account for the intelligibility of those performances.  When we
declare ``That's red'' in the presence of red things, we do so
``because'' (in some sense) of those red things and their (causal)
relation to us.  This is what Sellars dubbed a ``language entry''
move.  But that sort of causality cannot account for the conceptual
content of our utterance.

\subsubsection{Brandom: From Sentience to Sapience}

To say ``That's red'' is to apply the \textit{concept} ``red'', and
the subpersonal, causal relation between the presence of a red thing
and our conceptually contentful utterance cannot account for this.  It
cannot account for our ability to apply the concept red
\textit{correctly}, to red things, not non-red things.  After all, if
the presence of red things caused us to say ``That's red'', then we
would in fact say that hundreds or thousands of times a day.  A causal
model cannot account for four fundamental normative aspects of our
behavior: the ability to lie, to err, to hedge (``It \textit{seems}
red''), and to remain silent.

Brandom's Parrot: one of Brandom's favored illustrative examples is a
parrot trained to squawk ``That's red'' in the presence of red things.
This is an example of \textit{sentience} rather than
\textit{sapience}.  Brandom's Parrot is not sapient; its performance
does not count as conceptually contentful (rational), since it does
not involve the application of concepts.  This is where inferential
semantics enter the picture: the content of ``red'' is essentially
inferentially articulated.  To count as a concept user the parrot must
be capable of drawing inferences (either explicitly or implicitly)
involving the concept ``red''.  For example, it must know that
``That's green'' is incompatible with ``That's red''.  Those
inferences, in turn, are only intelligible in terms of what Brandom
(following Sellars) calls ``the game of giving and asking for
reasons''.

Question-based interviews: only intelligible as ``language games'',
denizens of the Space of Reasons.

\begin{remark}
  The fundamental mistake made by the \SM{} is failure to distinguish
  between distinct ``orders of explanation'': the subpersonal, causal
  world, and the personal, discursive, rational world.  Q\&A-based
  interviewing lives in the latter, not the former.  The notion that
  questions are stimuli that ``cause'' responses is fundamentally
  mistaken.  Whatever causal relations may obtain between a question
  utterance and the ensuing response utterance are not relevant to the
  intelligibility of the game.  Responses have \textit{reasons}, not
  causes.
\end{remark}

\begin{remark}
  An example would be useful here.  Maybe ``How old are you?''  A
  correct response to this question is one that involves propositional
  commitments and entitlements.  It does not involve any causal
  relationship to the question, still less to any ``latent'' age
  variable whose value is, say ``27 years''.  Crudely put, you know
  you're 27 years old if you know that last year you were 26.  More
  accurately, you know \textit{how} to respond because you know the
  rules of the language game, which involves also counting years and
  birthdays.  Consider how children learn their ages: they learn that
  certain verbal performances (e.g. ``I'm four'') are correct,
  regardless of whether they understand what they mean, and they learn
  that every year they have a ``birthday'', after which a different
  performance (``I'm five'') is correct.
\end{remark}

\subsection{Normativity}

The Space of Reasons is a normative space of reasons, not a natural
(causal) space of laws.

\subsection{Bibliography}

\noindent
\cite{bacon_pragmatism:_2012} \\
\cite{barnes_ethnomethodology_1985} \\
\cite{baert_pragmatism_2003} \\
\cite{baert_realism_2003} \\
\cite{baert_pragmatism_2004} \\
\cite{baert_towards_2005} \\
\cite{baert_philosophy_2005} \\
\cite{berard_rethinking_2005} \\
\cite{bloor_wittgenstein_2001} \\
\cite{blackburn_invited_1986} \\
\cite{blackburn_steps_2010} \\
\cite{brandom_mie} \\
\cite{brandom_precis_1997} \\
\cite{brandom_articulating_2001} \\
\cite{brandom_pragmatist_2004} \\
\cite{brandom_between_2008} \\
\cite{brandom_reason_2009} \\
\cite{brandom_perspectives_2011} \\
\cite{brandom_analyzing_2011} \\
\cite{brandom_classical_2011} \\
\cite{brandom_vocabularies_2011} \\
\cite{brandom_social_1993} \\
\cite{button_ethnomethodology_1991} \\
\cite{churchill_ethnomethodology_1971} \\
\cite{descombes_minds_2001} \\
\cite{emirbayer_pragmatism_2010} \\
\cite{garfinkel_studies_1984} \\
\cite{garfinkel_ethnomethodologys_2002} \\
\cite{heritage_garfinkel_1984} \\
\cite{kraut_varieties_1990} \\
\cite{loeffler_neo-pragmatist_2009} \\
\cite{lynch_ethnomethodology_2001} \\
\cite{lynch_cognitive_2006} \\
\cite{macdonald_nature_1992} \\
\cite{margolis_reinventing_2002} \\
\cite{margolis_pragmatism_2007} \\
\cite{maynard_diversity_1991} \\
\cite{maynard_toward_2000} \\
\cite{price_true_question_1983} \\
\cite{price_expressivism_2013} \\
\cite{price_naturalism_2013} \\
\cite{price_pluralism_2013} \\
\cite{price_two_2013} \\
\cite{putnam_representation_1991} \\
\cite{putnam_collapse_2002} \\
\cite{putnam_three_2009}
\cite{rorty_method_1981} \\
\cite{rorty_representation_1988} \\
\cite{rorty_PMN} \\
\cite{schatzki_practice_2001} \\
\cite{sellars_empiricism_1997} \\
\cite{tate_foucault_2007} \\
\cite{weiss_reading_2009} \\
\cite{winship_ethnomethodology_2010} \\
\cite{zimmerman_review_1994}

%%%%%%%%%%%%%%%%%%%%
\section{A Critique of the Theory of Cognitive Interviewing}

\begin{abstract}
  
\end{abstract}

%%%%%%%%%%%%%%%%%%%%
\section{Mensuration without Representation}

\begin{abstract}
Measurement pragmatism.  No representation needed.
\end{abstract}

\begin{remark}

Micro-macro:  temp as macro v. motion of molecules

Emergence: liquidity is an emergent property of H2O molecules; is temp
an emergent property of moving molecules?  It must be insofar as temp
is a subjective property (hot, cold, etc.)

Supervenience: or is temperature something that supervenes on groups
of molecules in motion?

\end{remark}

To measure is to characterize under a mathematical description.
Instead of ``measurement'', use the broader notion of mathematical
description.  So-called nominal measurement is not quantitative (nor
is ordinal measurement); calling it measurement clashes with our
intuition, which connects measurement with quantity or magnitude.  But
both do involve mathematical structure.  Mathematics is the science of
structure, not quantity.

Measurement claims are thus construed as claims about the structure of
some state of affairs in the world.  We express such claims in the
vocabulary of mathematics (plus an empirical vocabulary involving a
``dimension'' such as length); a ``valid'' measurement is a claim
expressing or describing a mathematical structure that corresponds
accurately (correctly) to the way things are.

Observable v. unobservable: implicit causal relationship.  Observable
as proxy for unobservable.  They must covary.

But this distinction is not simple.  Temperature \it{sensation} is
observable, but sensation is distinct from the property in the world.
When we measure temperature, we use proxy properties, such as the
height of a column of mercury.  So temperature is not observable in
the required sense.  That is, its mathematical structure is not
directly observable.  Contrast with measurement of length, which is
directly observable.  Or is it?  To measure length we rely on the
sensations involved in vision: we see that the measurand is twice the
length of the unit instrument.  But not really: we do not \it{see}
length per se; rather, we see a stick and use the term ``length'' to
express something about it, based on our experience with things in the
world, namely one of the ways we can compare them.  Which suggests
that terms like ``length'' are expressive in Brandom's sense: they
allow us to say what we can only otherwise do.  What we do is compare
things; saying that a stick is 1 meter long just saves us the trouble
of carrying out a comparison.

Alternatively we could express the same idea in terms of affordances:
when we look at a stick, we do not see its length, but we do see (so
to speak) one of its affordances: sticks afford lengthwise
comparison. (cf. Gibson)

Furthermore, there is the problem of the Myth of the Given and need to
explain how we go from merely responding to understanding.  This too
tends to subvert the observed/unobserved distinction, since we have to
ask just what it is that is observed, and what it is to observe.  We
cannot rely on mere sensory input, since that leads to the Myth.
Insofar as observation is a move in the Game of reasons, it is already
``theoretical'', that is, conceptual, from the very start.

IOW, the observable/unobservable distinction is often conflated with
the Given/theoretical distinction.  Observables are no more given than
unobservables are.  But they are directly connected to the causal
order.  So it would be better to talk of the distinction between
causal and rational orders instead of observables and unobservables.
Or perhaps we should stick to vocabulary talk, and make a distinction
between observation reports and other sorts of expressions.  Some
things afford observation reports, others do not.

Electrons are not observables; they do not afford observation reports.
But they are causally related to things that do afford such reports.
The job of theory is to articulate the hypothesized structure that
accounts for such reports in terms of causal relations with electrons.
This involves two of the three sorts of language moves: language
entries (things affording observation reports), and language-language
(theory).  Language exits involve what we do, not theoretical
predictions about what things in the world do, so the theory predicts
future language-entry moves (observations).

This is quite different from e.g. defining SES in terms of occupation,
etc.  Such definitions are conceptual and do not involve causal
relations.  Occupation does not cause SES; it is involved in what SES
\it{means} (inferentially), rather than what it is or how it came to
be.  So defining it is not not about discovering the nature of
something in the world.  Contrast definition of electron: it must
answer to the way things are in the causal order.  Our notion of SES
must only answer to the way things are in the normative order, which
is our order, our way of doing things, the way we cope.  If it's
useful, we use it; if not, we try other definitions.  There is no
question of its truth or accurate representation of something in the
world.  Its a piece of methodological pragmatism: its only purpose is
to explain our doings.  No metaphysics here, and also no (genuine)
measurement.  Putative measurements of SES should be treated as
methodological conveniences, not as claims about the true state of
affairs in the world.  Claims that may help us cope or decide what to
do, or even predict what will happen.  Not because we've measured some
fact in the causal order, but because we know something about norms,
and norms have a kind of predictive power.  We know what ought to be
the case; whether things in fact will turn out that way is a different
matter.

SES measures as descriptions, which do not necessarily entail
predictions.  Compare studies primate sociality.

Evolution, selective pressures, etc.  Primate anthropologists want to
discover selection pressures, not ``causes'' or the ordinary type.
That is causality in evolution is different than causality in physics.
Evolutionary causality v. nomological causality.  SES measures as a
way of getting at ``selection pressures'' that result in social
change, etc.

``We can use the kinds of methods described here to test hypotheses
about the selective forces that shape behavioral strategies and to
construct comparisons across individuals, groups, or taxa.'' (Silk
et al. p. 223)

\subsection{Previous Work}

``Paraphrasing N.R. Campbell (Final Report, p.340), we may say that
measurement, in the broadest sense, is defined as the assignment of
numerals to objects and events according to rules.''
\parencite[677]{stevens_theory_1946}


``[M]eaningful measurement is possible only if enough is known about
the attribute so as to justify its logical operationalization into
prescriptions from which a measurement instrument can be developed.''
\parencite[787]{sijtsma_psychological_2012}

I would rather say the measurement is possible only if we have a
theory of description that allows us to make predictions involving
measurable (observable) phenomena.

\subsection{Model Theory}

Truth and consequences and measurement claims.

Relevance of MT: (valid) measurement is all about representation,
reference, truth, and validity.  (Although a pragmatist might argue it
is about what works rather than what corresponds to reality.)
Tarski's semantic theory of truth and model-theoretic account of
consequence together form the pinacle of this approach.

Tarski (Convention T and model theory) as the pinacle of
representational accounts of truth and consequences.

Relevance to measurement?  We want to know if our measurement claims
are truth, and if the inferences we make involving such claims are
valid.

Measurement claims reduce to mathematical claims plus empirical
claims.  The mathematical part of this accounts for structure.

Model theory: to prove a logical consequence relation between a set of
statements $\Gamma$ and a statement $A$, first translate them from the
formal calculus to the language of ordinary theory (e.g. Group
Theory), and then prove the resulting theorems using the informal
techniques of the ordinary theory.

Is something similar involved in ``proving'' an empirical measurement
claim (which is a theory)?  One difference is goals: the goal of MT is
to show that the formal calculus is ``good''.  Science isn't too
worried about formal calculi, but it would presumably be a good thing
if we could express scientific theories formally and thereby enable
formal (automated) reasoning about them.  But we don't normally
express measurement claims in a formal calculus.  Indeed, since
measurement claims necessarily involve an empirical component
(e.g. units of measure involving empirical properties, that is
properties of things in the world), to do so would require formalizing
such empirical notions, thus draining them of their empirical
content).

%%%%%%%%%%%%%%%%
\subsection{Measurement as assignment of numbers}

``Paraphrasing N.R. Campbell (Final Report, p.340), we may say that
measurement, in the broadest sense, is defined as the assignment of
numerals to objects and events according to rules.'' (Stevens, 1946,
p.677).

This can't be entirely correct.  What we assign is not a numeral but a
location or position in a mathematical structure.  E.g. to assign '3'
to a quantity is not to attach a free-standing ``numeral'' to it, but
to assign it a place in the structure of integers.

So each scale type corresponds to a class of mathematical structures.

Nominal:  sets?  But sets are partially ordered.

Ordinal:  sets?  But sets also give us intervals?

A nominal scale seems to involve set membership (characteristic
functions) at least.  But if we can measure the size (cardinality) of
a set we end up with order and intervals.  So it looks like we must
stipulate that these mathematical properties are not to be ascribed to
the measurands.  Thus nominal measurement involves a partial mapping
to sets, or rather a mapping to a set structure that does not admit of
ordering or intervals.  Hmmm.

Ordinal scales involve order without difference.  Again that makes it
hard to think of ordinal measurement as involving mapping to sets.
Lattice theory?

Does it make sense to think of a mapping to a logical rather than a
mathematical structure?

Better: we take set theory a little bit at a time.  Start with the
basic axioms, then define preorders, posets, etc.  So we can treat
something as a poset without introducing cardinal and ordinal numbers
(I think).

In any case, the upshot is that (representational) measurement
postulates a mathematical structure to the measurand.

Michell's concern with whether or not a variable or construct is in
fact quantitative can be restated in structural terms.  Quantitative
properties etc. (in the world) have mathematical structure.  Or, to
say that something is measurable is to say that it has a particular
kind of structure.

Validity ``how well the measured variable represents the attribute
being measured'' comes out as \it{referential fidelity}.  Measurement
of something that lacks the requisite mathematical structure will then
lack referential fidelity.  Referential fidelity is broad enough to
cover both accuracy and precision of measurement.

\subsection{Validity as assessment of correctness}

I.e. to assess something as correct or incorrect is to measure it
against a norm.  In the case of e.g. temperature measurement, the norm
is the ``true'' temperature of the sample being measured.

Relevance: validity involves normativity and a kind of measurement
against (usually unstated) norms or ``true'' standards, which may be
(idealized) methods, etc.

Thus referential fidelity as correctness of representation.

\noindent References:

\noindent
\cite{chang_inventing_2004} \\
\cite{chang_measurement_2004} \\
\cite{chang_spirit_2004} \\
\cite{martin_counting_2009} \\
\cite{michell_normal_2000}\\
\cite{sherry_thermoscopes_2011}

See British Journal of Psychology, Aug 1997 vol 88 issue 3:
\cite{michell_quantitative_1997} and six commentaries.

%%%%%%%%%%%%%%%%
\subsection{Variables}

References:

\noindent
\cite{schwarz_is_2009}\\
\cite{toomela_variables_2008}\\
\cite{stam_fault_2010}

%%%%%%%%%%%%%%%%
\subsection{Error}

References:

\noindent
\cite{smith_refining_2011}

%%%%%%%%%%%%%%%%%%%%
\section{Deflating Validity}

\begin{abstract}
Semantic and metaphysical deflationism works as well for validity as
it does for truth and reference.
\end{abstract}

\begin{remark}
  Deflationism seems to depend essentially on some form of
  expressivism.  Or maybe they amount to the same thing?
\end{remark}

\subsection{Validity, Reliability, Error}
\label{sub:Validity}

\begin{remark}
What is the point of worrying about validity?  Is it something in the
world that we are trying to discover?  Then we're trying to find ``the
right description of the world'' (Putnam).  Or is it a concept, so
that validity talk is about conceptual analysis and definition?

Or: we try to find the right description, and validity talk is part of
how we decide that we have found it.

\end{remark}

\begin{remark}
Why do psychometricians and the like worry so about validity?

Hypothesis: when they say ``validity'', what they're really interested
in is scientific legitimacy.  Effectively, to say that a test (etc.)
is valid is to say that it is in fact scientific.  Thats the practical
import of the concept of validity for them.

Unpack this.  Expose the assumptions and implications.
\end{remark}

key concepts:

\begin{itemize}
\item validity treated as a special kind of property - of what?
\item constructs
\item (latent) variables
\item indicators
\end{itemize}

``validity'' as code for:

\begin{itemize}
\item legitimacy
\item vindication
\item credibility
\item proof (good premises + valid inference)
\end{itemize}

\begin{remark}
  On the idea that validity something (a property, etc.) that we look
  for in scientific theories in order to distinguish good ones from
  bad: see Putnam on fact/value distinction.  We use value judgments -
  simplicity, parsimony, etc. - in every aspect of science (thought),
  esp. in weeding out bad theories.  For there is no external or
  objective criterion of acceptability for theories to which we can
  appeal, nor is there any such citerion that does not involve value
  judgments.
\end{remark}

\begin{remark}
  So along with the fact/value distinction, and the analytic/synthetic
  distinction, the internal/external distinction also collapses?  Or
  do we just exclude the notion of external?  No; we need to retain
  the idea of an external world that is independent of us and to which
  some of our judgments are answerable.  We don't get to just make
  stuff up and call it true (correct) for at least some of our claims.
  There is no external absolute authority that can decide for us which
  theories are true, or rather which we should endorse, but that does
  not mean there is no external world that is authoritative for some
  of our sayings.  But isn't that trying to have it both ways?  How
  can our theories answer to the world if we cannot appeal to the
  world or some other external authority to sort them out?  See
  Brandom.

Related issue: what counts as evidence?  How do we decide?  What are
we doing when we decide that something counts as strong (weak)
evidence in support of a theory?  What are the criteria of adequacy
for an account of evidence?
\end{remark}

\subsection{RCT and Self-validation}

See Cartwright on RCT as self-validating.  This seems to mean that
RCTs are valid by construction.

This nicely parallels industrial QA notions of guaranteeing quality by
designing a production process that prevents defects.

What's the logic here?  Is self-validation really possible?  How can a
process validate itself - isn't the very idea inherently circular?  Or
rather, don't we land in a regress?  After all, if the idea is to
specify a process that yields validity, how do we know that that
process is itself valid?

\subsection{Deflation}

How can we get out of this mess?  One way is to deflate the notion of
validity, just deny that it is a substantive property.  When we claim
that a result is valid etc. what we are really saying is that we
endorse it, approve of it, etc.  It's an expressive device.  Compare
the semantic deflationist's idea that calling something true amounts
to endorsing or approving of it.

So if we discard the notion of validity (since it does no real work),
don't we find ourselves lacking something essential?  Well, we just
need a vocabulary that allows us to say explicitly the sorts of things
we find it useful to be able to express with respect to a study or qx
technique.  For example: credibility, utility, legitimacy,
vindication, justification, etc.

\begin{remark}
  The notion of validity seems to be connected to the problem of
  deciding which theories we should endorse.  What are the criteria of
  adequacy for any notion (or theory) of validity?  Or: what are the
  requirements that should be met by any purported explanation of
  validity?  Both particular cases and the general idea.  Tarski gives
  us something like this for logical validity; what about ``validity''
  as the term is used by psychometricians, test theorists, etc.?

Contrast: claims of validity for a case, v. explanation of what
validity is.


\end{remark}

The objection will no doubt be that we need some kind of standard,
which is just to say that we want to measure this something (validity,
credibility, whatever).  Implicit in all this is the notion that there
is some ``objective'' fact of the matter to which our
study/technique/etc. is ansswerable. A study is valid iff - what?  If
it meets some definite ``objective'' criteria.  Methodological
criteria, conditions of validity, etc.  In the psychometrics and
testing tradition this appeal to external authority is expressed as
something along the lines of ``measures what it purports to measure''.
Which is only meaningful insofar as a) there is actually something
there to measure, and b) it is in fact susceptibel to measurement.

And usually this is expressed in statistical terms.  But that dog
won't hunt either - you cannot get to validity via statistics.  All
you can do is measure central tendencies and variance - not enough to
establish validity, which is a substantive notion. (analysis
elsewhere).

To say that sth is valid is just to say that it is admirable
(Peirce?), or perhaps that it is virtuous, that it has the virtues we
prize.

%%%%%%%%%%%%%%%%%%%%%%%%
\section{The Deontic Scorekeeping Model of Discursive Practice and \SR{}}

\begin{abstract}
Why the deontic scorekeeping model is preferable to others, esp. the
cognitive model.
\end{abstract}

\begin{remark}
  It's a model of discursive, that is rational, practice.  Contrast
  this with most models on offer which tend to focus on subpersonal
  processes; hence the prevalence of talk about ``the survey
  process'', the ``response process'', etc.
\end{remark}

%%%%%%%%%%%%%%%%%%%%%%%%
\section{A Quality Assurance Model for \SR{}}

\begin{abstract}
abstract
\end{abstract}

%%%%%%%%%%%%%%%%%%%%%%%%
\section{Causality and the Space of Reasons}

\begin{abstract}
abstract
\end{abstract}

\noindent
\cite{abell_narrative_2004} \\
\cite{crane_mental_1995} \\
\cite{gross_pragmatist_2009} \\
\cite{jackson_mental_1996} \\
\cite{lowe_causal_1993} \\
\cite{lowe_non-cartesian_2006} \\
\cite{macdonald_mental_1986} \\
\cite{menzies_causation_1993} \\
\cite{morris_causes_1986} \\
\cite{williamson_broadness_1998}

\section{Vocabularies}

Measurement as description.  Description v. evaluation.  Price on
naturalisms.  The bifurcation thesis.

\section{Conflation of Causal and Logical Relations}



\section{Fact-Value}

Messick, for one, conflates two kinds of fact/value distinction.  The
Kantian idea that we structure our own experience (etc.), Sellars'
Myth of the Given, and etc. - such stuff shows how there is no data
that is ``objective'' and given i.e. ``data is theory-laden''.

So facts involve what Putnam calls ``epistemic values''.

Messick confuses epistemic and ethical values.  He seems to think that
although we cannot arrive at value-free facts, this is because brute
facts are always packaged with ethical values.  The idea seems to be
that ethical values are something separate from facts but always
attached to them somehow.  Whereas the real problem is that there is
no genuine distinction between fact and (epistemic) value.  Facts
express (as it were) our epistemic values.

Messick's confusion is clear in his distinction between the scientific
and social ``roles'' of validity - as if the social (value-laden)
aspect of (Messickian) validity is something distinct from the
science.  ``[I]t is fundamental that score validation is an empirical
evaluation of the meaning and consequences of measurement.  As such,
validation combines scientific inquiry with rational argument to
justify (or nullify) score interpretation and use.'' (p. 742) But
``scientific inquiry'' and ``rational argument'' are not two distinct
things that can be combined.  They are the same thing, at least
conceptually.  If there is a difference here, it is sociological -
science as a way of conduction oneself, etc.

Messickian validity boils down to some notion of empirical support for
theoretical explanations.  For him ``evidential basis'' seems to
correspond to ``real'' science, and ``consequential basis'' to
``rational argument''.

``[B]oth meaning and values are integral to the concept of
validity...'' (p. 747).  The problem here is that the contrast with
value is fact, not meaning.

``Meaning'' is not something that can be empirically ``validated''.

\section{Word-World}

One problem with e.g. Messick is fuzziness about the relation of
language to world.  Ditto for any notion of ``measuring a concept''.

Re: validity: is it supposed to be a property of things in the world,
or just a concept?  Per Messick, validity is ``associated with'' score
interpretation and use.  This would seem to imply that it is a matter
of language (concepts).  But the language is just sloppy; ``score
interpretation'' might (should) refer to how we take a score to relate
to some fact in the world, in which case the question is just what is
validity-in-the-world.

In any case, Messick's whole discussion is muddled on this point; it
is rarely clear when he is talking about facts, concepts, or the
relation between the two.  Is a ``construct'' supposed to be something
in the world or a concept the describes some aspect of the world?

Construct v. ``indicators''.

Compare positivist notions of observational language v. theoretical
language.  So-called indicators are (I understand) supposed to be
empirical observables.  Their relation to the construct is (must be) a
matter of theory; but then is that theoretical (conceptual) structure
to be taken as a mirror of reality, such that the construct is a real
(albeit ``hidden'') bit of the world and its relations to the
indicators are real relations in the world?

\section{Hypothetical Entities}

Putnam, Brandom, etc. - if the existence of (some) hypotheticals makes
no difference in the way things are then we can just discard them.  As
Putnam puts it, ``Would mathematics \it{work} one bit less well if
these funny objects \it{stopped} existing?  Those who posit ``abstract
entities''  to account for the success of mathematics do not claim that we (or any other things in the empirical world) \it{interact} with the abstract entities.  But if any entities do not interact with us or with the empirical world \it{at all}, then doesn't it follow that \it{everything wouuld be the same if they didn't exist}?'' (Collapse, p. 33)

This points out another problem with e.g. latent variables, namely
that they are supposed to have causal powers, but, insofar as they are
abstract at least, they have no connection to the empirical world and
so cannot cause anything.  The counterargument would presumably be
that hidden does not necessarily mean abstract.  But in that case they
must have a location in space-time, even if we don't know what it is.
But this just leads to more problems: where are hidden psychological
processes supposed to occur?  It can't be the brain, since they are
(by stipulation) psychological, not neurological.

So it seems we have no choice but to treat postulation of hidden stuff
as a matter of Brandomian methodological pragmatism: useful, but
without ontological consequences.  ``Constructs'' may be useful for
explaining observable indicators, but they don't really exist in any
meaningful sense.  But the usual story goes the other way around:
indicators are useful because they are how we get constructs.

\section{Personal v. Subpersonal}

Reasons v. causes

\section{Spaces}

\subsection{Natural space of causes}

\subsection{Discursive space of reasons}

\section{Notes}

\subsection{Evolution}

Instead of "the QA process", the proper object of investigation is the
local evolution of discourse.

EM studies local produced order.  It may come up with a structural
description.  But locally produced order is the outcome of an
essentially evolutionary process - the mutual adaptation of the
participants to each other and the context.  Also, any such model may
not (probably will not) generalize.  But what does generalize is the
evolutionary mechanism itself, just like in biology.

Rational selection as the mechanism of the evolution of discursive
performances.  What accounts for the deontic attitudes we adopt
regarding performances?  Brandom's account describes the architecture
of such posturings and the significances the institute.  But it does
not really address the logic of discourse as an evolutionary process.

The idea is that Brandom provides an account of discourse qua rational
action.  Different attitudes are endorsed or undertaken for reasons -
that is the source or ground of the intelligibility of discursive
practice.  So if we view the unfolding of discourse as being governed
by the logic of evolution, we can treat Brandom's sort of rational
pragmatism as the selection mechanism that accounts for why some
attitudes (meanings) survive (are endorsed) and others do not.
Meanings that survive must fit into the space of reasons - they must
be assertable and justifiable, even if the participants are unable to
explicitly articulate this.  This makes the evolution of discourse
intelligible as a rational process, rather than a natural process.
Responses to questions are not explicable as effects caused by "true
values" or the like; this would make them fundamentally non-rational.
Or to borrow a bon mot from Garfinkel, this would make respondents
"rational dopes".

Similar language: "negotiation", e.g. "...I suggest that the content
of talk indicates that imposed hierarchies are continually
re-negotiated..."  Negotiation as rational evolution?

The "true score" and other orthodox models account for sentience, not
sapience.

\subsection{Verum Factum}

Cartesianism (spectator, etc.) inspection, discovery, certainty,
foundationism (external foundation grounding knowledge) v.

Verum Factum, geneological/historical, following growth/development,
not certainty but ???; no foundationism, no priviledged vocab, no
external source of authority

Critical notions: authority.  For evidence etc. key idea is authority - the only
kind of authority is the kind we assent to.  So the question is what
do we treat as authoritative and why, rather than how can we discover
the One True external foundational source of authority and learn to
speak its language

Critical notions: vocabulary.  Regardless of what there is, we can
only talk about it by using vocabs.

Relevance to SR: we make our truths, by engaging in dialog with
respondents in order to teach/train them to understand what we want.
In other words we work to make our scorecards converge.  We can never
be sure that researchers and respondents understand each other, have
the same interpretations of qx text, etc.  But we can do what nature
does in evolution and learning: institute a cyclic process of
experiment, feedback, and correction.  This is operational even at the
most simple and basic level of communication.  So we can use this fact
to our advantage.

Communication interactions as not essentially different from processes
of evolution and learning.  Evolutionary process tend to coordinate
organism and environment; learning processes adapt the learner to the
task environment, etc.  Any discursive exchange - even simple
greetings, etc. - does the same sort of thing: coordinate and mutually
adjust the parties to the exchange.

\subsection{Rational Evidence}

Evidence-Based Rational SR

RCT: isolate the causal factor that links Treatment to Outcome

THe mistake make by orthodox SR (shown by its vocab of measurement,
error, etc.) is that it confuses the space of causes and the space of
reasons.

In RCT, we observe a stimulus followed by a response (T followed by O)
and postulate a causal relation.  In SR, we observe a Q performance
followed by a R performance.  In fact this is an idealization since Q
and R cannot be isolated - they are both joint performances.  Ignore
that for now; the point is that what makes them intelligible as
performances is the space of reasons, not causes.  That is, as
discursive episodes they are essentially rational in a way the T-O
trials are not.  By definition, "rational" means involving concepts.
Stimulus-response does not involve concepts and so is not rational in
this favored sense.  The natural world may be lawful, but it is not
rational.

So SR should abandon the orthodox vocab of measurment, etc. in favor
of one involving rationality.  What would "evidence-based" mean, then?
Not the kind of evidence involve in natural science, since such
evidence does not involve concepts and thus meaning.  Instead evidence
inescapably involves meaning and understanding.  What counts as
evidence is what we count as a rational explanation or story.  And
this necessarily involves the perspective of the participants - it is
their rationality, their giving and asking for reasons, that provides
the observational basis of evidence.

 One consequence: Qx does not involve measurement.  SR can use stats
 to statistically measure the collected data, but that is quite
 separate from whether the data measure anything.  So you can say that
 x\% of resondents pick option X, but that does not mean that you have
 measured the distribution of "true values" of some latent variable.
 What you have measure is a distribution of deontic scores, or
 discursive postures.  There is no warrant for claiming that each
 member of the x\% means the same thing by picking X.

\subsection{Misc}

1.  What is a question?  Better: what counts as a question, what is it to ask a question?

2.  Ditto for answer.

Q and A as parts of a whole (holistic view)

Q token v. Q performance, etc.

\subsection{Erotetic Discursive Practice}


EDP as production of data rather than discovery of truth

\subsection{Replication}

Goal is replication.  Compare: blood work, e.g. measuring
cholesteral.  The measuring apparatus reacts to the sample, not the
other way around.  For EDP, respondent reacts to the question, so the
question is analogous to the blood sample.  The response is a kind of
measurement of the question, not the other way around.

Replicability means same setup, same experimental conditions; in EDP
this means replication of conceptual structure, which is accomplished
by the dialog preceding the question.  Traditionally, "ask the same
question"; in practice this is impossible, since what counts is not
the question text but respondent's grasp of the sense.  So the
"experimental setup" should be viewed as the work of teaching the
respondent what the sense of the question is.  Survey interviewing is
essentially interventionist, but this is not necessarily a bad thing,
since lab experiments are too - they "intervene" to set up
experimental "initial conditions".  The difference is that setting up
initial conditions ("same meaning") in question asking means tutoring
the respondent.

\subsection{Myths and Mythologies}

\begin{itemize}
\item The Myth of Question Independence says that the meaning of a
    question is independent of context.  But the meaning of a question
    is always dependent on what came before it.
\item Myth of Autonomy. Interviewer and Respondent.
\item Myth of Error
\end{itemize}

\subsection{Dopes}

Garfinkel's dopes - cultural, judgmental, psychological

Dehumanization.  Orthodox Survey Research (OSR) dehumanizes
participants.  The R is a sampling unit.  The mythology of OSR
measurement treats the human R as a natural object to be measured
rather than a person.


\clearpage
\appendix
\begin{appendices}
\section{Bibliography}
%% \addcontentsline{toc}{chapter}{Bibliography}
%% \bibliographystyle{plainnat}
\printbibliography[heading=none]
\end{appendices}

\end{document}
